\begin{section}{Thermodynamic Calculation of an \acrNoHyperlink{\acs}{eos}}
% TODO Make clear what I have done and what is from other sources
This chapter aims at developing an fully special relativistic \ac{eos} of a noninteracting gas.
We briefly summarise important concepts necessary for the derivation.\\
In the canonical ensemble, for which a introduction can be found in \cite{fliessbachStatistischePhysikLehrbuch2018}, the internal energy $\U$ is obtained by the relation $\U=\F-TS$ where the quantities $\F$ and $S$ can be derived by means of the partition function $\Z$ while $T$ is a variable.
Microscopically, the partition function is given by the behaviour of the $N$ particles determined by the Hamiltonian $H$.
Concepts and definitions of Hamilton Mechanics can be found in \cite{eschrigTopologyGeometryPhysics2011, fliessbachMechanikLehrbuchZur2020, spivakPhysicsMathematiciansMechanics2010}.
In general we assume $H:\R^{3N}\times\R^{3N}\rightarrow\R$ to be a positive smooth function that assigns an energy to the positions and momenta of $N$ particles.
This explicitly takes form in the well known equations
\begin{align}
	\Z(T,V,N) 	&= \int\limits_{\mathcal{V}\times\R^N}\exp\left(-\frac{H(x_1,\dots,x_N,p_1,\dots,p_N)}{k_BT}\right)\frac{\diff x_1\diff p_1\dots \diff x_N\diff p_N}{N!h^{3N}}\\
	\F(T,V,N) 	&= - k_BT\log\left(\Z(T,V,N)\right)\label{2-IntEner-FreeEnerDef}\\
	\U(T,V,N) 	&= \F + T \Ent \hspace{0.2cm}=\hspace{0.2cm} \F - T\frac{\partial\F}{\partial T} \hspace{0.2cm}=\hspace{0.2cm} k_BT^2\frac{1}{\Z}\frac{\partial\Z}{\partial T}\label{2-IntEner-Def}
\end{align}
where $x_i\in \mathcal{V}\subset\R^3$  and $p_i\in \R^3$. The relation $V=\text{vol}(\mathcal{V})$ describes the total volume occupied by the medium while $T$ and $N$ are its temperature and particle number respectively.
To fully calculate $U$ it is necessary to obtain the partition function $\Z$.
From there $\U/V$ as the energy density can be compared to the pressure defined by
\begin{equation}
	p=\frac{\diff\F}{\diff V}
	\label{2-IntEner-pressureDef}
\end{equation}
to yield an \ac{eos}.
%
%
%
\begin{subsection}{The Ultra-Relativistic Internal Energy}
The ultra-relativistic case is a textbook example and well known limit that we will use in order to verify our later calculated results. 
First we write down the ultra-relativistic Hamiltonian given by 
\begin{equation}
	H(x,p)=||p||c.
\end{equation}
The corresponding partition function for a $N$ particle system then reads ($\beta=(k_BT)^{-1}$)
\begin{align}
	\Z 	&= \frac{V^N}{N!h^{3N}}\left[\int\limits_{\R^3}\exp\left(- \beta H(p)\right)\diff^3p\right]^N\\
		&= \frac{V^N}{N!h^{3N}}\left[\int\limits_0^\infty4\pi p^2\exp(-\beta pc) \diff p\right]^N\\
		&= \frac{V^N}{N!h^{3N}}\frac{(4\pi)^N}{(\beta c)^{3N}}\left[\int\limits_0^\infty x^2\exp(-x) \diff x\right]^N\label{2-IntEner-UltraRelZ-Int-eq}\\
	\Z	&= \frac{1}{N!}\left(8\pi V\left(\frac{k_BT}{hc}\right)^3\right)^N
	\label{2-IntEner-UltraRelZ}
\end{align}
where from the first to second line we used usual spherical coordinates and afterwards the integral transformation $x=\beta cp$.
The integral in equation \eqref{2-IntEner-UltraRelZ-Int-eq} can be solved exactly with value $2$. 
In the last line $\beta=(k_BT)^{-1}$ was used for visual clarity.
With equation \eqref{2-IntEner-Def} the internal energy $\U = 3Nk_BT$ and with equation \eqref{2-IntEner-pressureDef} the \ac{eos} can now be written down
\begin{equation}
	p = \frac{Nk_BT}{V} = \frac{1}{3}\frac{\U}{V} = \frac{1}{3}\rho.
	\label{2-IntEner-eos-nonrel}
\end{equation}
\end{subsection}
%
%
%
\begin{subsection}{The Special-Relativistic Internal Energy}
% We will now turn our attention to the fully special-relativistic case.
A source for the following calculations could not be found and were thus carried out by the author solely.
The special relativistic Hamiltonian is given by
\begin{equation}
	H=mc^2\sqrt{1+\frac{p^2}{m^2c^2}}.
\end{equation}
The ultra-relativistic limit can be obtained by letting $m\rightarrow0$. In this limiting case we should be able to recover the results from equation \eqref{2-IntEner-UltraRelZ}.
\begin{align}
	\Z 	&= \frac{V^N}{N!h^{3N}}\left[\int\limits_{\R^3}\exp\left(- \frac{mc^2\sqrt{1+\frac{p^2}{m^2c^2}}}{k_BT}\right)\diff^3p\right]^N\label{2-IntEner-EQ1}\\
		&= \frac{V^N}{N!h^{3N}}\left[\int\limits_0^\infty 4\pi p^2\exp\left(-\beta mc^2\sqrt{1+\frac{p^2}{m^2c^2}} \right)\diff p\right]^N\\
		&= \frac{(4\pi V)^N}{N!}\left(\frac{mc}{h}\right)^{3N}\left[\int\limits_0^\infty q^2\exp\left(-\alpha\sqrt{1+q^2}\right)\diff q\right]^N\\
		&= \frac{(4\pi V)^N}{N!}\left(\frac{mc}{h}\right)^{3N}\left(\int\limits_0^\infty\sinh(x)^2\cosh(x)\exp(-\alpha\cosh(x))\diff x \right)^N\\
		&= \frac{(4\pi V)^N}{N!}\left(\frac{mc}{h}\right)^{3N}\left(\int\limits_0^\infty\frac{\sinh(2x)}{2}\sinh(x)\exp(-\alpha\cosh(x))\diff x \right)^N\\
		&= \frac{(4\pi V)^N}{N!}\left(\frac{mc}{h}\right)^{3N}\Biggl(\left.-\frac{\sinh(2x)}{2\alpha}\exp(-\alpha\cosh(x))\right|_0^\infty\\
		&\hspace{12em} + \frac{1}{\alpha}\int\limits_0^\infty\cosh(2x)\exp(-\alpha\cosh(x))\diff x  \Biggr)^N\\
		&= \frac{(4\pi V)^N}{N!}\left(\frac{mc}{h}\right)^{3N}\left(\frac{1}{\alpha}\int\limits_0^\infty\cosh(2x)\exp(-\alpha\cosh(x))\diff x  \right)^N\\
		&= \frac{1}{N!}\left(8\pi V\left(\frac{k_BT}{hc}\right)^{3}\frac{\alpha^2K_2(\alpha)}{2}\right)^N
		\label{2-IntEner-PartFunc}
\end{align}
In the first step we used spherical coordinates followed by the substitution $qmc=p$ and $\alpha=\beta mc^2=mc^2/k_BT$. Afterwards we substituted $q=\sinh(x)$ and used the identity $\cosh(x)\sinh(x)=\sinh(2x)/2$. Partial integration then leads to the last integral which can be identified as the modified Bessel function of the 2nd kind $K_2(\alpha)$ \cite{abramowitzPocketbookMathematicalFunctions1984}. The equation is then rewritten such that the ultra-relativistic limit can be read off upon letting $\alpha\rightarrow0$.\\ % TODO find better formulation for "read off"
We can now calculate the internal energy $\U$ from $\Z$ via equation \eqref{2-IntEner-Def}
\begin{align}
    \U &= 3Nk_BT - Nk_BT\alpha\left(\frac{\partial_\alpha K_2(\alpha)}{K_2(\alpha)}+2\right)\\
    \U &= 3Nk_BT - mc^2\left(\frac{\partial_\alpha K_2(\alpha)}{K_2(\alpha)}+2\right).
    \label{2-IntEner-InternelEnergyExplicit}
\end{align}
Again, it can be seen that the ultra-relativistic limit can be obtained by letting $\alpha\rightarrow0$, since the term written in parenthesis vanishes.
It becomes immediately clear that the ultra-relativistic \ac{eos} is $\rho=3p$ which is equivalent to a traceless energy momentum tensor.\\
From equation \eqref{2-IntEner-PartFunc} and \eqref{2-IntEner-FreeEnerDef}, we immediately derive the ideal gas equation via the definition of pressure (see equation \eqref{2-IntEner-pressureDef} in the canonical ensemble
\begin{equation}
    p=\frac{\partial\F}{\partial V} = \frac{Nk_ BT}{V}.
    \label{2-IntEner-IdealGasEq}
\end{equation}
Note that in the case of the Hamiltonian $H$ having a non-trivial dependence on the coordinates $x_i$, the partition function would differ by a factor involving $V$ and $T$.
This more general scenario does not yield the ideal gas law and would thus need to be treated separately.
\end{subsection}
%
%
%
\begin{subsection}{The Special-Relativistic Equation of State}
\label{2-IntEner-SR-EOS-Derivation}
This section aims to develop an equation between the thermodynamic energy density $\rho=\U/V$ and the pressure $p$ of the gas given by the ideal gas equation \eqref{2-IntEner-pressureDef}.
We assume an additional constraint namely adiabaticity and thus further reduce the degrees of freedom of the thermodynamic system.\\ % TODO maybe different term for adiabaticity
When assuming an adiabatic condition $\updelta Q=0$ and using the First Law of Thermodynamics \cite{fliessbachStatistischePhysikLehrbuch2018} $\diff U =\updelta Q + \updelta W$, where $\updelta W=-p\diff V$ and $\diff U=C_V\diff T$, we can relate pressure and temperature. 
This adiabatic condition is quite well satisfied since the time scale of radiational and other losses compared to the time scale of thermodynamic events is negligible \cite{noerdlingerSolarMassLoss2008, vinkMassLossStellar2017}.
Using equation \eqref{2-IntEner-InternelEnergyExplicit} and \eqref{2-IntEner-pressureDef}, we obtain
\begin{alignat}{3}
    -p\diff V &= C_V\diff T\\
    -\frac{Nk_BT}{V}\diff V &= Nk_B &\Biggl[1 + &\alpha^2\left(\left(\frac{\partial_\alpha K_2(\alpha)}{K_2(\alpha)}\right)^2 - \frac{\partial^2_\alpha K_2(\alpha)}{K_2(\alpha)}\right)\Biggr]\diff T\\
    - \frac{\diff V}{V} &= &\Biggl[1 - &\alpha^2\partial_\alpha\left(\frac{\partial_\alpha K_2(\alpha)}{K_2(\alpha)}\right)\Biggr]\frac{\diff T}{T}\\
    &= &\Bigl[1 - &\alpha^2\partial_\alpha^2\left(\log K_2(\alpha)\right)\Bigr]\frac{\diff T}{T}
\end{alignat}
This equation also shows explicitly the $T$ dependence of the specific heat $C_V$ for the general case.
Again, taking the ultra-relativistic limit by letting $\alpha\rightarrow0$, the right-hand term in the first equation converges to $-2Nk_B$.
This agrees with the expected specific heat for an ultra-relativistic gas $C_{V,\text{ur}}=3Nk_B$.
With the identity $\diff\alpha/\alpha = -\diff T/T$ (using $\alpha=mc^2/k_BT$), we can transform the equation and integrate it.
After applying partial integration, the result is
\begin{align}
	\frac{\diff V}{V} &= \left(1-\alpha^2\partial_\alpha^2\log K_2(\alpha)\right)\frac{\diff\alpha}{\alpha}\\
    \log\left(\frac{V}{V_0}\right) &= \log\left(\frac{\alpha}{\alpha_0}\right) - \int\limits_{\alpha_0}^\alpha \alpha\frac{\partial^2}{\partial \alpha^2}\log(K_2(\alpha'))\diff\alpha'\\
    &=\log\left(\frac{\alpha}{\alpha_0}\right) + \log\left(\frac{K_2(\alpha)}{K_2(\alpha_0)}\right) - \Bigl[\alpha\frac{\partial_\alpha K_2}{K_2}\Bigr]_{\alpha_0}^\alpha
\end{align}
This equation enables us to write down a relation between volume and temperature (encapsulated in $\alpha=mc^2/k_BT$)
\begin{equation}
	V(\alpha) = \frac{\alpha K_2(\alpha)}{C}\exp\left(\alpha\frac{K_3(\alpha)+K_1(\alpha)}{2K_2(\alpha)}\right)
\end{equation}
where the constant $C$ is defined by the equation beforehand and only depends on the integration boundaries $\alpha_0$ and $V_0$. It is given by
\begin{equation}
	C = \frac{\alpha_0 K_2(\alpha_0)}{V_0}\exp\left(\alpha_0\frac{K_1(\alpha_0)+K_3(\alpha_0)}{2K_2(\alpha_0)}\right).
\end{equation}
Since the goal of this section is to obtain a readable output for an \ac{eos}, it is necessary to construct a bijection relating $p$ and $T$.
This becomes clear when writing down the energy density % TODO think about that formulation
\begin{equation}
	\rho = \frac{\U}{V} = \frac{Nk_BT}{V} - \frac{Nk_BT}{V}\left(\alpha\frac{\partial_\alpha K_2(\alpha)}{K_2(\alpha)}\right)
	\label{2-IntEner-DensityAlpha}
\end{equation}
where $p=Nk_BT/V$ can be easily identified but the $T$ dependence via $\alpha$ is not solved yet.\\
The pressure $p$ can be rewritten to take the form
\begin{equation}
	p = \frac{Nk_BT}{V} = CNmc^2\frac{1}{K_2(\alpha)\alpha^2}\exp\left(-\alpha\frac{K_1(\alpha)+K_3(\alpha)}{2K_2(\alpha)}\right).
	\label{2-IntEner-PressureAlpha}
\end{equation}
At this point it is not reasonable to ask what happens in the ultra-relativistic limit since $C$ depends non-trivially on $m$ and thus $m$ is not fully substituted in $\alpha$.\\
Interestingly, the pressure seems to be constant for very high temperatures.
The limiting case is obtained when taking $T\rightarrow\infty$ (which corresponds to $\alpha\rightarrow0$)
\begin{equation}
	\lim\limits_{\alpha\to0}\left[\frac{1}{K_2(\alpha)\alpha^2}\exp\left(-\alpha\frac{K_1(\alpha)+K_3(\alpha)}{2K_2(\alpha)}\right)\right] = \frac{1}{2\e^2} \approx 0.006767.
	\label{2-IntEner-PressureAlpha-Limit}
\end{equation}
The same argument then holds true for the density given by equation \eqref{2-IntEner-DensityAlpha} and since 
\begin{equation}
	\lim\limits_{\alpha\to0}\left[1+\alpha\frac{K_1(\alpha)+K_3(\alpha)}{2K_2(\alpha)}\right] = 3
\end{equation}
we have
\begin{equation}
	\lim\limits_{\alpha\to0}\left(\frac{\rho(\alpha)}{Cmc^2}\right) = \frac{3}{2\e^2} \approx 0.203003.
\end{equation}
%
%
\begin{theorem}
	The mapping $p:\R_{>0}\rightarrow \R_{>0},\alpha\mapsto p(\alpha)$ written down in equation \eqref{2-IntEner-PressureAlpha} is a bijection for any $N,m,c,C\neq0$.
	\begin{equation}
		p = \frac{Nk_BT}{V} = CNmc^2\frac{1}{K_2(\alpha)\alpha^2}\exp\left(-\alpha\frac{K_1(\alpha)+K_3(\alpha)}{2K_2(\alpha)}\right)
	\end{equation}
\end{theorem}
\begin{proof}
	For this proof it suffices to show that the function $p(\alpha)$ has a strictly monotonous behaviour.
	Without loss of generality, we assume $N,m,c,C>0$.
	Now it is obvious that the first two terms $Nmc^2/CK_2(\alpha)$ and $\alpha^{-2}$ are strictly decreasing.
	This is easy to see when using \cite{abramowitzPocketbookMathematicalFunctions1984}
	\begin{equation}
		\frac{\partial K_n}{\partial\alpha} = \frac{n}{\alpha}K_n-K_{n+1} = -\frac{K_{n-1}+K_{n+1}}{2}.
		\label{2-IntEner-BesselDerivative}
	\end{equation}
	We then calculate the derivative of the third term and divide by the exponential (since it is positive)
	\begin{align}
		&\hspace{1em} \frac{1}{\exp(\alpha\partial_\alpha\log(K_2))}\frac{\partial}{\partial\alpha}\exp(\alpha\partial_\alpha\log(K_2))\\
		&= \hspace{1em} \partial_\alpha\log(K_2) + \alpha\partial^2_\alpha\log(K_2)\\
		&= \hspace{1em} \frac{\partial_\alpha K_2}{K_2} + \alpha\frac{\partial^2_\alpha K_2}{K_2} + \alpha\left(\frac{\partial_\alpha K_2}{K_2}\right)^2\\
		&= \hspace{1em} \frac{K_1-\frac{2}{\alpha}K_2}{K_2} + \alpha\frac{\frac{1}{\alpha}K_1-K_2-\frac{2}{\alpha}\left(K_1-\frac{2}{\alpha}K_2\right)+\frac{2}{\alpha^2}K_2}{K_2^2} + \alpha\frac{K_1^2-\frac{1}{\alpha}K_1K_2-\frac{4}{\alpha^2}K_2^2}{K_2^2}\\
		&= -\alpha + \alpha\frac{K_1^2}{K_2^2} - 4\frac{K_1}{K_2}
	\end{align}
	thus, it is sufficient to show that
	\begin{equation}
		\alpha\frac{K_1^2}{K_2^2}  < \alpha + 4\frac{K_1}{K_2}.
	\end{equation}
	We quickly prove the more general result $K_\nu<K_{\nu+1}$.
	One possible definition \cite{abramowitzPocketbookMathematicalFunctions1984} for the Bessel function $K_\nu$ is given by
	\begin{equation}
		K_\nu := \frac{\sqrt{\pi}}{\left(\nu-\frac{1}{2}\right)!}\left(\frac{1}{2}z\right)^\nu\int\limits_1^\infty\e^{-tz}\left(t^2-1\right)^{\nu-\frac{1}{2}}\diff t
	\end{equation}
	We inspect the ratio
	\begin{equation}
		\frac{K_\nu}{K_{\nu+1}} = \frac{1}{\left(\nu+\frac{1}{2}\right)\left(\frac{1}{2}z\right)}\frac{\int\e^{-tz}\left(t^2-1\right)^{\nu-1/2}\diff t}{\int\e^{-tz}\left(t^2-1\right)^{\nu+1/2}\diff t}
	\end{equation}
	and rewrite the demoninator with partial integration
	\begin{equation}
		 \frac{1}{2}z\int\limits_1^\infty\e^{-tz}\left(t^2-1\right)^{\nu+1/2}\diff t = \left(\nu+\frac{1}{2}\right)\int\limits_1^\infty \e^{-tz}t\left(t^2-1\right)^{\nu-1/2}\diff t.
	\end{equation}
	Now it is obvious that $K_{\nu+1}>K_\nu$.
	Thus in total, the function given by equation \eqref{2-IntEner-FinalEOS} can be inverted.
\end{proof}%
%
%
\noindent With this mapping $p:\R_{>0}\rightarrow \R_{>0},\alpha\mapsto p(\alpha)$ and its inverse $\alpha:\R_{>0}\rightarrow \R_{>0},p\mapsto \alpha(p)$, we can use \eqref{2-IntEner-DensityAlpha} and finally write down the \ac{eos} 
\begin{equation}
	\rho = \frac{\U}{V} = p\left(1+\alpha(p)\frac{K_1(\alpha(p))+K_3(\alpha(p))}{2K_2(\alpha(p))}\right).
	\label{2-IntEner-FinalEOS}
\end{equation}
Figure \ref{2-IntEner-RelEOSPlot} is obtained by choosing numerical values and then interpolating and inverting equation \eqref{2-IntEner-PressureAlpha}.
The constant factor
\begin{equation}
	B:=CNmc^2/p_0
	\label{2-IntEner-FactorExplanation}
\end{equation}
is substituted to obtain independence of $p_0$\footnote{This will be the starting value for the pressure in the \acl{TOV} equation. Since the pressure decreases from the inside a star to the outside, this resembles our highest value of $p$.}.
Furthermore the graphs of the plotted \ac{eos} are normalised such that
\begin{equation}
	\rho_{i}(p_0)=\rho_{0,i}
\end{equation}
and thus can be compared with each other.
Note that the $\rho_0$ of the plot is not a universal value but rather each function has been scaled individually and $\rho_0$ is a placeholder for the corresponding $\rho_{0,i}$.
The polytropic \ac{eos} $\rho_{cla}(p)=Ap^{1/\gamma}$ when normalised as before, the equation is independent of $A$ and can be uniquely characterised by $n$. % TODO is this comma right?
For further details, the interested reader is referred to \cite{pleyerGithubRepositoryJonas2021}.
\begin{figure}[H]
	\centering
	\import{pictures/2-InternalEnergy-Explicit/}{RelEOS.pgf}
	\caption[Relativistic Equation of State]{Relativistic Equation of State -
	The relativistic \ac{eos} $\rho(p)$ is normalised such that values can be compared with a polytropic \ac{eos}.
	Graphs for the relativistic version are independent of the exponent $n$ which is a degree of freedom intrinsic to the polytropic \ac{eos}.
	By normalisation, the graphs of the polytropic \ac{eos} are independent of the factor $A$.}
	\label{2-IntEner-RelEOSPlot}
	% TODO general formatting of figures. Keep description in lower font and maybe center title above?
\end{figure}
\end{subsection}
%
%
%
% \begin{subsection}
% TODO Generalise the approach given above for a interacting gas under certain conditions.
% \end{subsection}
%
%
%
\end{section}
