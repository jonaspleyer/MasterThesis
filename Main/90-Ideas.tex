\section{Ideas}
\begin{theorem}[Lane-Emden Finite Boundary]
	\label{Theo-Lane-EmdenFiniteBoundary}
	For every $0\leq n\leq5$, the Lane-Emden equation
	\begin{equation}
		\frac{1}{\xi^2}\frac{d}{d\xi}\left(\xi^2\frac{d\theta}{d\xi}\right)+\theta^n=0
		\label{5-Ideas-LEeq}
	\end{equation}
	with $\theta_0=1$ has a zero value for a finite $\xi_0$. To prove this theorem, show the statements \ref{Theo-Lane-EmdenFiniteBoundary1}, 
	\ref{Theo-Lane-EmdenFiniteBoundary2} and \ref{Theo-Lane-EmdenFiniteBoundary3}.
\end{theorem}
\begin{theorem}[Lane-Emden Local Existence]
	\label{Theo-Lane-EmdenFiniteBoundary1}
	For every $n\geq0$, the Lane-Emden equation \ref{5-Ideas-LEeq} has a solution in an $\epsilon$ Ball around $\xi=0$.
\end{theorem}
\begin{proof}
	This proof can be generalised to $m\geq2$ dimensions. 
	Following \cite{quittnerSuperlinearParabolicProblems2007a} we see that the LE differential equation can be written as
	\begin{equation}
		\frac{d}{d\xi}\left(\xi^{m-1}\frac{d\theta}{d\xi}\right) + \xi^{m-1}\theta^n = 0.
	\end{equation}
	Integrating once and rearranging leads us to
	\begin{equation}
		\frac{d\theta}{d\xi} = \left.\frac{d\theta}{d\xi}\right|_0 - \int\limits_0^\xi\left(\frac{t}{\xi}\right)^{m-1}\theta^ndt.
	\end{equation}
	By choosing the initial value of $d\theta/d\xi=0$ we see that positive solutions to the LE equation need to have negative slope.
	This equation also shows that for $\theta_0>0$, the solution $\theta$ will be positive in a $\epsilon$ region around $\xi=0$.
	We integrate both sides again and end up with
	\begin{equation}
		\theta = \theta_0 - \int\limits_0^\xi\int\limits_0^s\left(\frac{t}{s}\right)^{m-1}\theta^ndtds.
		\label{5-Ideas-LEOperator}
	\end{equation}
	We define $C^*$ as the space of all real continuous functions $f\geq0$ on $[0,\epsilon]$ that share the same initial value $\theta_0$ and with negative slope. %where $n\in\R$ refers to the exponent given before.
	With the norm $||.||_\infty$ this space is complete.
	Inspecting the operator $T:C^*\rightarrow C^*$ defined by
% 	\footnote{The  index $1$ refers to the $C^1$ norm given by $||f||_2=||f||_\infty+||f'||_\infty+||f''||_\infty$ on the yet to be defined space.}
	equation \ref{5-Ideas-LEOperator} we can apply the Banach fixed-point theorem.
% 	Without loss of generality, we can assume $U\neq V$. Then
	\begin{align}
		||T(U)-T(V)||_\infty	&\leq	\left|\left|U^n-V^n\right|\right|_\infty\int\limits_0^\xi\int\limits_0^s \left(\frac{t}{s}\right)^{m-1}dtds\\
								&=		\left|\left|U^n-V^n\right|\right|_\infty\frac{\xi^2}{2m}
	\end{align}
	Let $n\geq1$. 
	Since the function $x\mapsto x^n$ is differentiable for $n\geq1$ it is also Lipschitz with constant $L$ on the compact interval $I=[0,\theta_0]$.
	This means in particular that we have
	\begin{equation}
		||T(U)-T(V)||_\infty	\leq	\left|\left|U-V\right|\right|_\infty\frac{L\xi^2}{2m}
	\end{equation}
	Thus when choosing $\epsilon<\sqrt{2m/L}$ we have a contraction and the Banach fixed-point theorem then gives us a unique solution to equation \ref{5-Ideas-LEOperator}.\\
	Now let $0\leq n<1$.
	In this situation the preceding method does not work due to $x^n$ not being differentiable at $x=0$. 
	For $\theta_0\leq1$, we know that $\theta^n\leq\theta_0^n$. We can now choose $\epsilon$ such that with
	\begin{equation}
		\int\limits_0^\xi\int\limits_0^s\left(\frac{t}{s}\right)^{m-1}\theta^n(t)dtds \leq \frac{\xi^2}{2m}\theta_0^n
	\end{equation}
	we have $\theta\geq\theta_0/2$ on $[0,\epsilon]$. If $\theta_0>1$, we have $\theta^n\leq\theta_0$ and obtain the same result. 
	Then the same procedure as before can be carried through but with the interval $I=[\theta_0/2,\theta]$.
\end{proof}
%
%
%
\begin{theorem}[Lane-Emden Finite Boundary 2]
	\label{Theo-Lane-EmdenFiniteBoundary2}
	For $n<5$, the Lane-Emden equation \ref{5-Ideas-LEeq} has no Solution defined on the total space $\R_{\geq0}$.
\end{theorem}
\begin{proof}
	From \cite{quittnerSuperlinearParabolicProblems2007a} we know that solutions for equation \ref{5-Ideas-LEeq} on the total space exist for $n\geq5$.
	
\end{proof}
%
%
%
\begin{theorem}[Lane-Emden Finite Boundary 3]
	\label{Theo-Lane-EmdenFiniteBoundary3}
	Let $n<5$ and $\theta:[0,x)\rightarrow\R$ be a solution of \ref{5-Ideas-LEeq}. If $\theta(x)>0$, then $\theta$ can be extended with Theorem \ref{Theo-Lane-EmdenFiniteBoundary1} 
	to $[0,x+\epsilon)$ where $\epsilon$ follows a growth condition $\epsilon(\xi)=...$ such that if $\theta$ would have no zero point, there would be a solution on total 
	$\R_{\geq0}$ and thus a contradiction with \ref{Theo-Lane-EmdenFiniteBoundary2}.
\end{theorem}
\begin{theorem}[Lane-Emden Finite Boundary 4]
	The Lane Emden equation has a zero value if the exponent function $n(\xi)$ fullfills the growth condition XYZ.
\end{theorem}

\begin{theorem}[TOV Exact Solution]
	The TOV equation \ref{4-NumSol-TOVEqBasic1} with a polytropic EOS $\rho=Ap^{1/\gamma}$ has a well defined limiting case where $A\rightarrow0$ with $m=0$ and
	\begin{equation}
		p = \frac{p_0}{2\pi rp_0+1}
	\end{equation}
\end{theorem}
\begin{proof}
	First we transform the TOV equation \ref{3-Mass-TOV-Eq} using $p=y^a$ with $a>0$ and $m=Arv$ and together with the polytropic EOS $\rho=Ap^{1/\gamma}$ obtain
	\begin{align}
		\frac{\partial v}{\partial r} &= 4\pi ry^{a/\gamma}-Av\\
		ay^{a-1}\frac{\partial y}{\partial r} &= -\frac{vA^2y^{a/\gamma}}{r}\left(1+\frac{y^a}{Ay^{a/\gamma}}\right)\left(\frac{4\pi r^2y^a}{Av} +1\right)\frac{1}{1-2vA}
		\label{tmp-label-2}
	\end{align}
	Rearranging the second equation one obtains
	\begin{equation}
		\frac{\partial y}{\partial r} = -\frac{y^{a/\gamma-a+1}}{ar}\left(A+y^{a-a/\gamma}\right)\left(4\pi r^2y^a +Av\right)\frac{1}{1-2vA}.
	\end{equation}
	Using $\gamma=1+1/n$ with $n>0$, we see that this equation is continuous in every variable $(r,y,v)\in\R_{>0}\times\R_{\geq0}\times[0,1/2A)$. We restrict 
	ourselves to a compact domain $r\in \overline{B_{\delta}(\tau)}$ and $v\in[0,1/2A-\epsilon]$ where $0<\epsilon<1/2A$. 
	We choose $0<\delta$ and $0\leq\tau$ in such a way that $0<\epsilon<1/2A$ is satisfied for some $\epsilon$ which is possible since $m\rightarrow0$ for $r\rightarrow0$.
	To obtain Lipschitz continuity in $(y,v)$, all of the following conditions need to be fulfilled.
	\begin{equation}
		\frac{a}{\gamma}-a+1 \geq 1 \hspace{1cm} a-\frac{a}{\gamma} \geq 1 \hspace{1cm} a \geq 1
		\label{tmp-label-1}
		% TODO Change label when the proof is complete and in the correct section
	\end{equation}
	The second equation implies the first and the third. Thus we only need to choose $a\geq(1-1/\gamma)^{-1}$. With $\gamma=1+1/n$ we can rewrite this equation to 
	\begin{equation}
		a\geq n+1
	\end{equation}
	which can be easily satisfied. 
	This now shows with extension of the Picard-Lindelöf Theorem that there exists a unique solution for given initial values $\tau, y(\tau), v(\tau)\in B_{\delta}(\tau)\times\R_{\geq0}\times[0,1/2A-\epsilon)$ that especially continuously depends on $A$.
	Without loss of generality we can choose $\delta$ small enough such that solutions are positive in $p$.
	%Let $\epsilon\rightarrow0$ and $\delta\rightarrow0$ be two (wlog monotonous) sequences with identical initial values $\tau, y(\tau), v(\tau)$. Respective solutions are identical for 
	%all values on which both are defined. This means we can extend our solution to cover $(0,\infty)\times(0,\infty)\times[0,1/2A)$. 
	By transforming back equation \ref{tmp-label-2} to 
	\begin{equation}
		\frac{\partial p}{\partial r} = -\frac{1}{r^2}\left(Ap^{1/\gamma}+p\right)\left(4\pi r^3p+vA\right)\left(1-2vA\right)^{-1}
	\end{equation}
	and letting $A\rightarrow0$, we obtain
	\begin{equation}
		\frac{\partial p}{\partial r} = - 4\pi rp^2
	\end{equation}
	which is then solved by
	\begin{equation}
		p = \frac{\tilde{p}}{2\pi\tilde{p}(r^2-\tau^2)+1}
	\end{equation}
	where $\tilde{p}$ is the initial value at $r=\tau$. It is clear that this solution can be extended to $r\in[0,\infty)$ by choosing arbitrary small $\tau$.
	In the limit, the result equals the hypothesis.
\end{proof}
\begin{theorem}[TOV Finite Boundary]
	For each $A>0$, the TOV equation \ref{4-NumSol-TOVEqBasic1} has a exponent $n>0$ for which a solution $p$ does not have any zero points.
\end{theorem}
\begin{theorem}[TOV Finite Boundary 1]
	If $p_1$ and $p_2$ are two solutions of the TOV equation (for equal $A$) with $p_i$ corresponding to $n_i$ exponents, then if $n_1<n_2$, the zero 
	point (if it exists) of $p_1$ is smaller than that of $p_2$ (if it exists or is $\infty$).
\end{theorem}
\begin{theorem}[TOV Finite Boundary 2]
	a 
\end{theorem}




