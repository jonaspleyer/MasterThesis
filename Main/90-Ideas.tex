\section{Ideas}

%
%
%
\begin{theorem}[Lane-Emden Finite Boundary 2]
	\label{5-Zeroes-Lane-EmdenFiniteBoundary2}
	For $n<5$, the Lane-Emden equation \ref{5-Ideas-LEeq} has no Solution defined on the total space $\R_{\geq0}$.
\end{theorem}
\begin{proof}
	From \cite{quittnerSuperlinearParabolicProblems2007a} we know that solutions for equation \ref{5-Ideas-LEeq} on the total space exist for $n\geq5$.
	
\end{proof}
%
%
%
\begin{theorem}[Lane-Emden Finite Boundary 3]
	\label{5-Zeroes-Lane-EmdenFiniteBoundary3}
	Let $n<5$ and $\theta:[0,x)\rightarrow\R$ be a solution of \ref{5-Ideas-LEeq}. If $\theta(x)>0$, then $\theta$ can be extended with Theorem \ref{5-Zeroes-Lem-Lane-EmdenFiniteBoundary1} 
	to $[0,x+\epsilon)$ where $\epsilon$ follows a growth condition $\epsilon(\xi)=...$ such that if $\theta$ would have no zero point, there would be a solution on total 
	$\R_{\geq0}$ and thus a contradiction with \ref{5-Zeroes-Lane-EmdenFiniteBoundary2}.
\end{theorem}
\begin{theorem}[Lane-Emden Finite Boundary 4]
	The Lane Emden equation has a zero value if the exponent function $n(\xi)$ fullfills the growth condition XYZ.
\end{theorem}


\begin{theorem}[TOV Finite Boundary]
	For each $A>0$, the TOV equation \ref{4-NumSol-Equ-TOVEqBasic1} has a exponent $n>0$ for which a solution $p$ does not have any zero points.
\end{theorem}
\begin{theorem}[TOV Finite Boundary 1]
	If $p_1$ and $p_2$ are two solutions of the TOV equation (for equal $A$) with $p_i$ corresponding to $n_i$ exponents, then if $n_1<n_2$, the zero 
	point (if it exists) of $p_1$ is smaller than that of $p_2$ (if it exists or is $\infty$).
\end{theorem}
\begin{theorem}[TOV Finite Boundary 2]
	a 
\end{theorem}




