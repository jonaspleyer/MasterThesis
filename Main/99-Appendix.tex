\begin{appendix}
\pagenumbering{roman}
\renewcommand{\thesection}{\Alph{section}}
\renewcommand{\thesubsection}{\Alph{subsection}}

\begin{section}*{Appendix}
\addcontentsline{toc}{section}{Appendix}
%
%
\begin{subsection}{Exact solutions of the LE equation}
\label{99-App-A-Exact-LE-Solutions}
This section partly relies on information in \cite{weissteinLaneEmdenDifferentialEquation} and \cite{chandrasekharChandrasekharAnIntroductionStudy1958}. The LE equation is
\begin{equation}
	\frac{1}{\xi^2}\frac{d}{d\xi}\left(\xi^2\frac{d\theta}{d\xi}\right)+\theta^n=0
	\label{99-App-A-LE-Equation}
\end{equation}
which for $n=0$ transforms readily into
\begin{equation}
	\int\limits_0^\xi\frac{d}{d\xi}\left(\xi'^2\frac{d\theta}{d\xi}\right)d\xi' = -\int\limits_0^\xi\xi'^2d\xi'.
\end{equation}
Both sides can be evaluated directly and then simplified further.
\begin{align}
	\xi^2\frac{d\theta}{d\xi} &= -\frac{\xi^3}{3}\\
	\frac{d\theta}{d\xi} &= -\frac{\xi}{3}\\
	\theta(\xi) &= \theta(0)-\frac{\xi^2}{6}
\end{align}
With the initial condition $\theta(0)=1$, we obtain the desired result. For $n=1$, equation \ref{99-App-A-LE-Equation} transforms into
\begin{equation}
	\frac{d}{d\xi}\left(\xi^2\frac{d\theta}{d\xi}\right)+\xi^2\theta=0
\end{equation}
which is a spherical Bessel differential equation
\begin{equation}
	\frac{d}{dr}\left(r^2\frac{dU}{dr}\right)+\left[k^2r^2-m(m+1)\right]U=0
\end{equation}
when setting $m=0$ and $k=1$. The solution to this equation is \cite{weissteinSphericalBesselDifferential}
\begin{align}
	U(r) 	&= A\sqrt{\frac{\pi}{kr}}J_{m+1/2}(kr) + B\sqrt{\frac{\pi}{kr}}Y_{m+1/2}(kr)\\
			&= Aj_m(kr) + By_m(kr)
\end{align}
where $J_n$ is the Bessel function of the first and $Y_n$ of the second kind and $j_n$ and $y_n$ are the corresponding spherical Bessel functions. Thus in our case with $j_{0}(x)=\sin(x)/x$ and $Y_{0}(x)=-\cos(z)/z$, we obtain
\begin{equation}
	\theta(\xi) = A\frac{\sin(\xi)}{\xi} - B\frac{\cos(k\xi)}{\xi}.
\end{equation}
The need for a well defined limit at $\xi\rightarrow0$ implies that $B=0$ and thus since $\sin(z)/z\rightarrow1$ for $z\rightarrow0$, we have $A=\theta(0)=1$ and 
\begin{equation}
	\theta(\xi) = \frac{\sin(\xi)}{\xi}.
\end{equation}
For $n=5$
% TODO Add n=5 exact results





\end{subsection}
%
%
\end{section}
\end{appendix}
