\begin{section}{Numerical Solutions}
\begin{subsection}{Comparing TOV and LE results with a polytropic EOS}
In this section the numerical solutions of the TOV equation
\begin{align}
	\frac{\partial m}{\partial r} &= 4\pi\rho r^2\label{4-NumSol-TOVEqBasic1}\\
	\frac{\partial p}{\partial r} &=-\frac{m\rho}{r^2}\left(1+\frac{p}{\rho}\right)\left(\frac{4\pi r^3p}{m}+1\right)\left(1-\frac{2m}{r}\right)^{-1}
	\label{4-NumSol-TOVEqBasic2}
\end{align}
as derived previously in \ref{3-Mass-TOVDerivation} will be discussed. To obtain numerical solvability a equation of state in the form $\rho(r,p)$ is supplied. 
% TODO Add reference for when the equations are solvable
In Figure \ref{4-NumSol-TOVEqEasyEOS} a plot of such a solution is presented.
The density $\rho$ is derived via the equation \ref{4-NumSol-TOVEqBasic1} and the integration is done with a 4th order Runge-Kutta Method. The integration is stopped once the pressure reaches values $p\leq0$. The initial value of \ref{4-NumSol-TOVEqBasic1} is $\partial m/\partial r(r=0)=0$. For \ref{4-NumSol-TOVEqBasic1}, the initial value can be calculated when applying L'Hôpital's and combining them to obtain
\begin{alignat}{3}
	\lim\limits_{r\rightarrow0}\frac{m}{r} &= \lim\limits_{r\rightarrow0}\frac{\partial m}{\partial r} &=\lim\limits_{r\rightarrow0}\frac{4\pi\rho r^2}{1} &= 0\\
	\lim\limits_{r\rightarrow0}\frac{m}{r^2} &= \lim\limits_{r\rightarrow0}\frac{1}{2r}\frac{\partial m}{\partial r}  &= \lim\limits_{r\rightarrow0}\frac{4\pi\rho r^2}{2r} &= 0\\
	\lim\limits_{r\rightarrow0}\frac{m}{r^3} &= \lim\limits_{r\rightarrow0}\frac{1}{3r^2}\frac{\partial m}{\partial r} &=\lim\limits_{r\rightarrow0}\frac{4\pi\rho r^2}{3r^2} &= \frac{4\pi\rho_0}{3}\\
	\lim\limits_{r\rightarrow0}\frac{\partial p}{\partial r} &= 0
\end{alignat}
Explicit code can be found in \cite{pleyerGithubRepositoryJonas}. The Lane-Emden equation was obtained in section \ref{3-Mass-LEDerivation} as the non-relativistic limit of the TOV equation by neglecting terms of order $1/c^2$ and higher and setting $G=c=1$. Equation \ref{3-Mass-Lane-Emden-Eq} was used to obtain numerical values for the Lane-Emden equation. Afterwards the factor $\alpha$ was reinserted with the corresponding numerical values given to the TOV equation to obtain comparable results.
\begin{equation}
	\frac{1}{\xi^2}\frac{d}{d\xi}\left(\xi^2\frac{d\theta}{d\xi}\right) + \theta^n = 0
	\label{4-Num-Sol-LE}
\end{equation}
one could be tempted to make the assumption that
$\frac{\partial p_{\text{TOV}}}{dr} \leq \frac{\partial p_{\text{LE}}}{dr}.$
However this equation fails since the mass given in equation \ref{4-NumSol-TOVEqBasic1} is not the same as the one given in the LE equation. Figure \ref{4-NumSol-TOVEqEasyEOS} shows the solution of both equations for the parameters of table \ref{4-NumSol-TOVParameters}. Note that values in the last part are calculated instead of supplied to the solving routine. Additionally conversion equations to compare TOV and LE results are displayed.\\
Since the mass reads
\begin{equation}
	m(r) = \int\limits_0^r 4\pi s^2\rho(s)ds,
\end{equation}
we expect $\partial m/\partial r(R)=0$ if $p(R)=0$ when choosing a polytropic equation of state with $\gamma>0$. The plot in figure \ref{4-NumSol-TOVEqEasyEOS} shows this expected behaviour for the Lane Emden equation at $r\approx2.31$ and has the same behaviour for the TOV results at $r\approx6.80$.\footnote{For the purpose of nicely displaying the calculated result, the plot only shows result up to $r=2.5$}
\begin{table}[H]
	\renewcommand{\arraystretch}{1.2}
	\centering
	\begin{tabular}{@{}llcll@{}}
		\toprule
		\multicolumn{2}{c}{\textbf{TOV}} & \phantom{abc} &\multicolumn{2}{c}{\textbf{LE}}\\
		\cmidrule{1-2} \cmidrule{4-5}
		EOS & $\rho=Ap^{1/\gamma}$ && EOS & $p=K\rho^{\gamma}$\\
		$A$ & $2$ & & \\
		$\gamma=1+\frac{1}{n}$ & $4/3$ && $n=1/(\gamma-1)$ & $3$\\
		$p_0$ & $0.5$ && $\theta_0$ & $1$\\
		$m_0$ & $0$ && $d\theta_0$ & $0$\\
		$dr$ & $0.01$ && $d\xi=dr/\alpha$ & $0.01/0.3355\approx0.0298$\\
		\cmidrule{1-2} \cmidrule{4-5}
		$\rho_0=Ap_0^{1/\gamma}$ & $2(2)^{\frac{4}{3}}\approx1.1892$ && $\lambda=\rho_0$ & $2(2)^{\frac{4}{3}}\approx1.1892$\\
		&&& $K=A^{-1/\gamma}$ & $2^{-3/4}\approx0.5946$\\
		&&& $\alpha^2=((n+1)K\lambda^{1/n-1})/(4\pi)$ & $\approx0.1125$\\
		\bottomrule
	\end{tabular}
	\caption[Numerical Parameters for TOV and Lane-Emden equation]{Parameters for numerical solving of the TOV and Lane-Emden equation.}
	\label{4-NumSol-TOVParameters}
\end{table}%
% In this particular case $1<\gamma=4/3$ and thus the slope of the TOV-density does not necessarily have to be smaller. By taking the derivative of the density with respect to the equation of state one sees that
% \begin{equation}
% 	\frac{\partial\rho}{\partial r} = \frac{\rho^{\gamma^{-1}-1}}{C^{1/\gamma}}\frac{\partial p}{\partial r}
% \end{equation}
% and thus if $\rho$ has small enough values that the slope of the TOV-density may falls below the Lane-Emden solution. The plot of this particular case shows this small detail at the last part of the plotted interval.
\begin{figure}[H]
	\centering
	\import{pictures/4-NumericalSolutions/}{TOV-LE-Combi.pgf}
	\caption[Comparison TOV and LE equation]{Comparison of the fully relativistic TOV result and the classical Lane-Emden solution. The images show the plots for the parameters of table \ref{4-NumSol-TOVParameters}. First the pressure is presented. Afterwards the density calculated with the given EOS and the average density $\bar{\rho}_i=(4\pi/3)^{-1}m_i/r^3$ for the two solutions are beeing compared. In the second row, the mass and the ratio $m_i/r$ can be seen.}
	\label{4-NumSol-TOVEqEasyEOS}
\end{figure}
\end{subsection}
%
%
\begin{subsection}{Verifying the results}
Verification is done in different ways. First, one can compare calculated LE results with already known exact  solutions for certain exponents as given in table \ref{3-Mass-LE-Exact-Results}. All calculations are carried out with a chosen stepsize of $d\xi=0.03$.
% TODO Make better validation plot with d\xi=0.03 and insert it here.
\begin{figure}[H]
	\centering
	\import{pictures/4-NumericalSolutions/}{LE-ValidateSols.pgf}
	\caption[Validation of numerical LE results]{The plot shows the exact solutions for the LE equation as given by table \ref{3-Mass-LE-Exact-Results}. Afterwards the difference of the exact to the numerically calculated result is shown. The y-scale of each plot is multiplied by a factor of $10^{-5}$ which shows that the solutions agree up to very high precision.}
	\label{4-NumSol-ValidateLEResults}
\end{figure}

% TODO TOV-Terms: Why do we do this? Explain it better.
To verify the equality of both solutions, we calculate the LE result with the TOV solving algorithm by dropping terms from right to left in equation \ref{4-NumSol-TOVEqBasic2}. These intermediate solutions have been numerically calculated and results can be seen in figure \ref{4-NumSol-TovTerms}.
\begin{figure}[H]
	\centering
	\import{pictures/4-NumericalSolutions/}{TOV-Terms.pgf}
	\caption[Comparison LE and partial TOV]{Comparison between the LE and TOV solutions while dropping terms from equation \ref{4-NumSol-TOVEqBasic2} from right to left. The last figure shows the difference between the TOV solution with 0 terms and the LE solution. The scale of the difference shows that the numerical differ only by values up to $3.5\times10^{-7}$. In order to achieve such a comparison, a polynomial fit of both pressure solutions had to be done. This should however not alter the result in any noticeable way. For further details see \cite{pleyerGithubRepositoryJonas}.}
	\label{4-NumSol-TovTerms}
\end{figure}

\end{subsection}
%
%
\begin{subsection}{Relativistic EOS}
In the previous discussion, we relied on the EOS given by \ref{4-NumSol-TOVEqEasyEOS}. This is a versatile assumption, but one could ask, what would happen to a star in which the particles have no interaction but are near relativistic speed. The resulting EOS was calculated in the beginning \ref{2-IntEner-FinalEOS} although not written down explicitly. Since explicit inversion of the given function is hard, we rely on numerical methods for calculation.
% TODO TOV-Rel: Explain and create plot to put here.
\end{subsection}
%
%
\end{section}
