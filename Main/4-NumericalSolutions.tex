\begin{section}{Numerical Solutions}
\begin{subsection}{Polytropic EOS}
In this section the numerical solutions of the TOV equation
\begin{align}
	\frac{\partial m}{\partial r} &= 4\pi\rho r^2\label{4-NumSol-TOVEqBasic1}\\
	\frac{\partial p}{\partial r} &=-\frac{m\rho}{r^2}\left(1+\frac{p}{\rho}\right)\left(\frac{4\pi r^3p}{m}+1\right)\left(1-\frac{2m}{r}\right)^{-1}
	\label{4-NumSol-TOVEqBasic2}
\end{align}
as derived previously in \ref{3-Mass-TOVDerivation} will be discussed. To obtain numerical solvability a equation of state in the form $\rho(r,p)$ is supplied. 
% TODO Add reference for when the equations are solvable
In Figure \ref{4-NumSol-TOVEqEasyEOS} a plot of such a solution is presented.
The density $\rho$ is derived via the equation \ref{4-NumSol-TOVEqBasic1} and the integration is done with a 4th order Runge-Kutta Method. The integration is stopped once the pressure reaches values $p<=0$.  Explicit code can be found on the github profile of the author\cite{pleyerGithubRepositoryJonas}. Upon comparing the TOV equation and its non-relativistic limiting case, the Lane-Emden equation\footnote{Neglecting terms of order $1/c^2$ and higher and setting $G=c=1$.} as derived in \ref{3-Mass-LEDerivation},
\begin{equation}
	\frac{\partial p}{\partial r} = -\frac{m\rho}{r^2}
\end{equation}
one promptly notices that
\begin{equation}
	\frac{\partial p_{\text{TOV}}}{dr} \leq \frac{\partial p_{\text{LE}}}{dr}.
\end{equation}
This means that if the Lane-Emden equation has value of $p_\text{LE}=0$, the TOV equation will also have one, and if $p_\text{LE}(0)\neq0$ then $r_\text{0,TOV}<r_\text{0,LE}$. However depending on the equation of state, this may not be said with regard to the density $\rho_\text{TOV}$ as can be seen in the numerical example below. Figure \ref{4-NumSol-TOVEqEasyEOS} shows the solution of both equations for the parameters:
\begin{table}[h]
	\centering
	\begin{tabular}{|c|c|}
		\hline
		Parameters & Values \\
		\hline
		Equation of state & $\rho=Ap^{1/\gamma}$ \\
		$A$ & $2$\\
		$\gamma$ & $4/3$ \\
		$p_0$ & $1$ \\
		$\rho_0=Ap_0^{1/\gamma}$ & $2$\\ 
		$m_0$ & $0$\\
		\hline
	\end{tabular}
	\caption[Numerical Parameters for TOV and Lane-Emden equation]{Parameters for numerical solving of the TOV and Lane-Emden equation.}
	\label{4-NumSol-TOVParameters}
\end{table}
In this particular case $1<\gamma=4/3$ and thus the slope of the TOV-density does not necessarily have to be smaller. By taking the derivative of the density with respect to the equation of state one sees that
\begin{equation}
	\frac{\partial\rho}{\partial r} = \frac{\rho^{\gamma^{-1}-1}}{C^{1/\gamma}}\frac{\partial p}{\partial r}
\end{equation}
and thus if $\rho$ has small enough values that the slope of the TOV-density may falls below the Lane-Emden solution. The plot of this particular case shows this small detail at the last part of the plotted interval.
\begin{figure}[H]
	\centering
	\includesvg[width=\textwidth]{pictures/4-NumericalSolutions/TOV-LE-Combi}
	\caption[Comparison TOV and Lane-Emden equation]{Comparison of the fully relativistic TOV result and the classical Lane-Emden solution. The images show the plots for the parameters of table \ref{4-NumSol-TOVParameters}}
	\label{4-NumSol-TOVEqEasyEOS}
\end{figure}\noindent
Generally speaking this result shows the expected behaviour. One has to keep in mind that due to numerical errors, the point at which the respective solutions for $p$ cross the $x$-axis can not be determined perfectly. This issue is especially prominent for the TOV equation, since the function tends towards $0$ very slowly.\\
A manifest of this behaviour can also be seen at the end of the slope of the TOV-Mass $m_{\text{TOV}}$. Since
\begin{equation}
	m(r) = \int\limits_0^r 4\pi s^2\rho(s)ds,
\end{equation}
we expect $\partial m/\partial r(R)=0$ for the given polytropic equation of state \ref{4-NumSol-TOVParameters} with parameter $\gamma>0$. The plot does not perfectly show this expected behaviour which can again be traced back to numerical uncertainties as discussed earlier.
%Since the metric outside the star is expected to have the Schwarzschild metric, we must demand $m(r)=M=\text{const}$ for $r\geq R$. And since the derivative of the mass vanishes at $R$, we have a (at least) $C^1$ mass function $m(r)$ which is indeed required by assumption. By equation \ref{3-Mass-EinstEqu-2}, we see that the pressure needs to be at least continuous to yield a $C^1$ metric. This is indeed satisfied if $p(R)=0$. Additionally from equation \ref{4-NumSol-TOVEqBasic2}, we can see that $\partial p/\partial r(R)=0$ (again for the polytropic equation of state) which shows that $\e^\nu$ is $C^2$ at $R$ and thus since solutions of \ref{4-NumSol-TOVEqBasic2} are smooth on $(0,R)$, $\e^\nu$ is at least $C^2$ on $\R_{>0}$.
A cross check to test if the limiting case as calculated in section \ref{3-Mass-LEDerivation} by dropping all additional terms in equation \ref{4-NumSol-TOVEqBasic2} yields good results. Additionally, when dropping terms from right to left in equation \ref{4-NumSol-TOVEqBasic2} more intermediate solutions can be numerically calculated. Results can be seen in figure \ref{4-NumSol-TovTerms}.
\begin{figure}[ht]
	\centering
	\includesvg[width=\textwidth]{pictures/4-NumericalSolutions/TOV-Terms}
	\caption[Comparison LE and partial TOV]{Comparison between the LE and TOV solutions while dropping terms from equation \ref{4-NumSol-TOVEqBasic2} from right to left. The last figure shows the difference between the TOV solution with 0 terms and the LE solution. The scale of the difference shows that the numerical differ only by values up to $3.5\times10^{-7}$. In order to achieve such a comparison, a polynomial fit of both pressure solutions had to be done. This should however not alter the result in any noticeable way. For further details see \cite{pleyerGithubRepositoryJonas}.}
	\label{4-NumSol-TovTerms}
\end{figure}

\end{subsection}
%
%
\begin{subsection}{Relativistic EOS}
In the previous discussion, we relied on the EOS given by \ref{4-NumSol-TOVEqEasyEOS}. This is a versatile assumption, but one could ask, what would happen to a star in which the particles have no interaction but are near relativistic speed. The resulting EOS was calculated in the beginning \ref{2-IntEner-FinalEOS} although not written down explicitly. Since explicit inversion of the given function is hard, we rely on numerical methods for calculation.

\end{subsection}
%
%
\end{section}
