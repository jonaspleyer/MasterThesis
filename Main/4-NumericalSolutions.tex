\begin{section}{Numerical Solutions}
\begin{subsection}{Comparing \texorpdfstring{\acrNoHyperlink{\acs}{TOV}}{TOV} and \texorpdfstring{\acrNoHyperlink{\acs}{LE}}{LE} results with a polytropic \texorpdfstring{\acrNoHyperlink{\acs}{eos}}{EoS}}
\label{subsec:4-NumSol-Sec-Comp-TOV-LE}
In this section the numerical solutions of the \ac{TOV} equation
\begin{align}
	\frac{\partial m}{\partial r} &= 4\pi\rho r^2\label{eq:4-NumSol-Equ-TOVEqBasic1}\\
	\frac{\partial p}{\partial r} &=-\frac{m\rho}{r^2}\left(1+\frac{p}{\rho}\right)\left(\frac{4\pi r^3 p}{m}+1\right)\left(1-\frac{2m}{r}\right)^{-1}
	\label{eq:4-NumSol-Equ-TOVEqBasic2}
\end{align}
as derived previously in section~\ref{subsec:3-Mass-Sec-TOVDerivation} will be discussed.
To obtain numerical solvability an \ac{eos} in the form $\rho(r,p)$ is supplied.
In Figure~\ref{fig:4-NumSol-Plt-TOVEqEasyEOS}, a plot of such a solution is presented.
The density $\rho$ is derived via the equation~\eqref{eq:4-NumSol-Equ-TOVEqBasic1} and the integration is done with a 4th order Runge-Kutta Method~\cite{rungeUeberNumerischeAufloesung1895, schlömilch1901zeitschrift, h.SimplifiedDerivationAnalysis2010}.
The integration is stopped once the pressure reaches values $p\leq0$.
The initial value of~\eqref{eq:4-NumSol-Equ-TOVEqBasic1} is $\partial_r m(r=0)=0$.
For equation~\eqref{eq:4-NumSol-Equ-TOVEqBasic1}, the initial value can be calculated when applying L'Hôpital's rule and combining them to obtain
\begin{alignat}{5}
	\lim\limits_{r\rightarrow0}\frac{m}{r} &= \lim\limits_{r\rightarrow0}\frac{\partial m}{\partial r} &&=\lim\limits_{r\rightarrow0}\frac{4\pi\rho r^2}{1} &&= 0\\
	\lim\limits_{r\rightarrow0}\frac{m}{r^2} &= \lim\limits_{r\rightarrow0}\frac{1}{2r}\frac{\partial m}{\partial r}  &&= \lim\limits_{r\rightarrow0}\frac{4\pi\rho r^2}{2r} &&= 0\\
	\lim\limits_{r\rightarrow0}\frac{m}{r^3} &= \lim\limits_{r\rightarrow0}\frac{1}{3r^2}\frac{\partial m}{\partial r} &&=\lim\limits_{r\rightarrow0}\frac{4\pi\rho r^2}{3r^2} &&= \frac{4\pi\rho_0}{3}\\
	\lim\limits_{r\rightarrow0}\frac{\partial p}{\partial r} &= 0
\end{alignat}
Explicit code can be found in~\cite{pleyerGithubRepositoryJonas2021}.
The Lane-Emden equation was obtained in section~\ref{subsec:3-Mass-Sec-LEDerivation} as the non-relativistic limit of the \ac{TOV} equation by neglecting terms of order $1/c^2$ and higher and setting $G=c=1$.
To obtain numerical results for the \ac{LE} equation as given in equation~\eqref{eq:3-Mass-Equ-Lane-Emden-Eq}, a substitution of the form $d\theta/d\xi=\chi$ was used.
\begin{equation}
	\begin{aligned}
		\frac{d\theta}{d\xi} &= \chi &\hspace{1cm} \left.\frac{d\theta}{d\xi}\right|_{\xi=0} &= 0\\
		\frac{d\chi}{d\xi} &= -\frac{2}{\xi}\chi-\theta^n & \left.\frac{d\chi}{d\xi}\right|_{\xi=0} &= -\theta^n
		\label{eq:4-NumSol-Equ-LE-Substitution}
	\end{aligned}
\end{equation}
The initial value for $d\chi/d\xi$ can be calculated with L'Hôpital's rule.
With the conversion factor $\kappa$ derived in section~\ref{subsec:3-Mass-Sec-LEDerivation}, \ac{TOV} and \ac{LE} results can be compared.
Figure~\ref{fig:4-NumSol-Plt-TOVEqEasyEOS} shows the solution of both equations for the parameters of Table~\ref{tab:4-NumSol-Tbl-TOVParameters}.
Additionally conversion equations to compare \ac{TOV} and \ac{LE} results are displayed.
Since the mass reads
\[
	m(r) = \int\limits_0^r 4\pi r'^2\rho(r')dr',
\]
we expect $\partial m/\partial r(R)=0$ if $p(R)=0$ when choosing a polytropic equation of state with $\gamma>0$.
The plot in Figure~\ref{fig:4-NumSol-Plt-TOVEqEasyEOS} shows this expected behaviour for the Lane Emden equation at $r\approx2.31$ and has the same behaviour for the \ac{TOV} results at $r\approx6.80$.\footnote{For the purpose of nicely displaying the calculated result, the plot only shows result up to $r=2.5$}
\begin{table}[H]
	{\renewcommand{\arraystretch}{1.2}
	\centering
	\begin{tabular}{@{}llcll@{}}
		\toprule
		\multicolumn{2}{c}{\textbf{TOV}} & \phantom{b} &\multicolumn{2}{c}{\textbf{LE}}\\
		\cmidrule{1-2} \cmidrule{4-5}
		EOS & $\rho=Ap^{1/\gamma}$ && EOS & $p=K\rho^{\gamma}$\\
		$A$ & $2$ & & \\
		$\gamma=1+\frac{1}{n}$ & $4/3$ && $n=1/(\gamma-1)$ & $3$\\
		$p_0$ & $0.5$ && $\theta_0$ & $1$\\
		$m_0$ & $0$ && $d\theta_0$ & $0$\\
		$dr$ & $0.01$ && $d\xi=dr/\kappa$ & $\approx0.0298$\\
		\cmidrule{1-2} \cmidrule{4-5}
		$\rho_0=Ap_0^{1/\gamma}$ & $2(2)^{\frac{4}{3}}\approx1.1892$ && $\lambda=\rho_0$ & $2(2)^{\frac{4}{3}}\approx1.1892$\\
		&&& $K=A^{-1/\gamma}$ & $2^{-3/4}\approx0.5946$\\
		&&& $\kappa^2=((n+1)K\lambda^{1/n-1})/(4\pi)$ & $\approx0.1125$\\
		\bottomrule
	\end{tabular}
	}
	\caption[Numerical Parameters for \acrNoHyperlink{\acs}{TOV} and LE equation]{Numerical Parameters for \ac{TOV} and \ac{LE} equation}
	\label{tab:4-NumSol-Tbl-TOVParameters}
	\small
	To compare results, values $\rho_0,K,\kappa$ are calculated from ones supplied to the solving routine.
\end{table}%
\begin{figure}[H]
	{\centering
	\import{pictures/4-NumericalSolutions/}{TOV-LE-Combi.pgf}
	}%
	\caption[Comparison of the \acrNoHyperlink{\acs}{TOV} and \acrNoHyperlink{\acs}{LE} equation]{Comparison of the \ac{TOV} and \ac{LE} equation.}
	\label{fig:4-NumSol-Plt-TOVEqEasyEOS}
	\small
	The images show the plots for the parameters of Table~\ref{tab:4-NumSol-Tbl-TOVParameters}.
	The pressures and masses as presented in the first column are direct solutions of the \ac{TOV} or \ac{LE} equations.
	On the top right-hand side the densities calculated with the \ac{eos} and average densities $\bar{\rho}_i=(4\pi/3)^{-1}m_i/r^3$ are being compared.
%	First the pressure is presented.
%	Afterwards the density calculated with the given \ac{eos} and the average density $\bar{\rho}_i=(4\pi/3)^{-1}m_i/r^3$ for the two solutions are being compared.
%	In the second row, the masses and ratios $m_i/r$ can be seen.
\end{figure}
\end{subsection}
%
%
\begin{subsection}{Verifying the results}
\label{subsec:4-NumSol-Sec-Verifiying-the-results}
% Verification is done in different ways. 
% First, o
One can compare calculated \ac{LE} results with already known exact  solutions for certain exponents as given in table~\ref{fig:3-Mass-Tbl-LE-Exact-Results}.
The graphs of figure~\ref{fig:4-NumSol-Plt-ValidateLEResults} overall show good numerical agreement.
The integration stepsize for the left-hand side results of figure~\ref{fig:4-NumSol-Plt-ValidateLEResults} is $d\xi=0.03$.
Initial values at $r=0$ are identical since they were set to be.
The spike occurring afterwards can be explained by equations~\eqref{eq:4-NumSol-Equ-LE-Substitution}.
The initial value of all derivatives is identically $0$ which means no change in the values $\theta$ or $\chi$ for the initial step of the numerical integration.
On the other hand the exact results will show a change which explains the large discrepancy in the beginning.
%It is clear that the relative difference on the right-hand side will spike again once smaller values for $\theta$ are reached if the absolute difference stays constant.
%This behaviour is best seen in the solution for $n=5$ where a large range of $0\leq r\leq100$ for solving was chosen.
%The offset between the numerically calculated values and the exact result stays constant while the value of $\theta$ decreases inverse proportionally.
Despite this behaviour we can see that the criterion for convergence is well fulfilled in the right-hand side of figure~\ref{fig:4-NumSol-Plt-ValidateLEResults}.
The plot displays nicely that by lowering the integration stepsize, the maximum difference in this interval decreases.
\begin{figure}[H]
	\centering
	\import{pictures/4-NumericalSolutions/}{LE-ValidateSols-2.pgf}
	% TODO If time left fix left axis of this plot
	\caption{Validation of numerical LE results}
	The first two plots show absolute and relative difference of known exact and numerically calculated results while the third plot shows the behaviour of the maximum value of $\Delta$ as the stepsize decreases.
	\label{fig:4-NumSol-Plt-ValidateLEResults}
\end{figure}\noindent
%As can be seen in the last column of plots, the condition for convergence in this case is well fulfilled.
% TODO TOV-Terms: Why do we do this? Explain it better.
% Next, we wish to see how accurate numerical \ac{TOV} results are. This is not an easy task since no explicit solutions are known for the \ac{TOV} equation. However one way to at least get confident that solving was a success is by dropping terms from right to left in equation~\eqref{4-NumSol-Equ-TOVEqBasic2} and solving for these new differential equations. As we have seen in section~\ref{3-Mass-LEDerivation}, the last result should reproduce the \ac{LE} equation. These intermediate differential equations have been numerically solved and results can be seen in Figure~\ref{4-NumSol-TovTerms}.
% \begin{figure}[H]
% 	\centering
% 	\import{pictures/4-NumericalSolutions/}{TOV-Terms.pgf}
% 	\caption[Comparison of \ac{LE} and partial TOV solutions]{Comparison of \ac{LE} and partial \ac{TOV} solutions - The index $i$ represents the number of terms still present in the equation where $i=3$ means the complete \ac{TOV} equation and $i=0$ the \ac{LE} equation. The last figure shows the difference between the \ac{TOV} solution with $0$ terms and the \ac{LE} solution calculated via equation~\eqref{4-NumSol-Equ-LE-Substitution}. The scale of the difference shows a maximum of $\approx3.5\times10^{-7}$. In order to achieve such a comparison, a polynomial fit of both pressure solutions had to be done. This should however not alter the result in any noticeable way. For further details see~\cite{pleyerGithubRepositoryJonas}.}
% 	\label{4-NumSol-TovTerms}
% \end{figure}
% 
\end{subsection}
%
%
\begin{subsection}{Relativistic \texorpdfstring{\acrNoHyperlink{\acs}{eos}}{EoS}}
\label{subsec:4-NumSol-Sec-RelEOS}
In the previous discussion, we relied on the \ac{eos} given in Table~\ref{tab:4-NumSol-Tbl-TOVParameters}.
This is a versatile assumption, but one could ask what would happen to a star in which the particles have no interaction but are close to relativistic speed.
The resulting \ac{eos} was calculated in the beginning (see equation~\eqref{eq:2-IntEner-FinalEOS}) although not written down explicitly.
Since explicit inversion of the given function is complicated, we rely on numerical methods for calculation.
Section~\ref{2-IntEner-SR-EOS-Derivation} showed that the function given in equation~\eqref{eq:2-IntEner-PressureAlpha}
\begin{equation}
	p(\alpha) = CNmc^2\frac{1}{K_2(\alpha)\alpha^2}\exp\left(-\alpha\frac{K_1(\alpha)+K_3(\alpha)}{2K_2(\alpha)}\right).
	\label{eq:4-NumSol-Pressure-Alpha-Dep}
\end{equation}
explicitly has a non-positive slope. 
When calculating an inverse function the values $C,N,m$ need to be chosen high enough in order to reach sufficiently high values for $p$.
By knowing the initial value $p_0$ and by the fact that the pressure of the \ac{TOV} solution~\eqref{eq:3-Mass-Equ-TOV-Eq-1} has a non-positive slope and with equation~\eqref{eq:2-IntEner-PressureAlpha-Limit}, it is sufficient to choose $Bp_0=CNmc^2>2/\e^2$.
On the other hand it is also important to choose high enough values for $\alpha$ for numerical inversion since otherwise functions can not be displayed if they reach small enough values for $p$.
The limit for $\alpha$ is chosen such that values as low as $0.001p_0$ can be evaluated.
Afterwards the function over the given interval is inverted by a polynomial fit and then used in equation~\eqref{eq:2-IntEner-DensityAlpha} to obtain the \ac{eos}.\\
Figure~\ref{fig:4-NumSol-Plt-RelEOS-TOV-Comparison} is an example which shows a clear difference between the polytropic and relativistic \ac{eos}.
The values used are $B=20$, $p_0=0.4$, $dr=0.001$, $n=1$.
The correct prefactor $A$ of the polytropic \ac{eos} is chosen after numerical calulation of the relativistic \ac{eos}.
It is then simply given by 
\begin{equation}
	A=\frac{\rho_\mathrm{rel}(p_0)}{p_0^{1/\gamma}}.
	\label{eq:4-NumSol-EOS-Factor-Explanation}
\end{equation}
This normalises both \acp{eos} to same inital pressures.
\begin{figure}
	\centering
	\import{pictures/4-NumericalSolutions}{TOV-RelEOS-Comparison.pgf}
	\caption[Comparison of TOV polytropic and relativistic EOS]{Comparison of TOV polytropic and relativistic \acs{eos}}
	\label{fig:4-NumSol-Plt-RelEOS-TOV-Comparison}
\end{figure}
\end{subsection}
%
%
\begin{subsection}{Zero Values of the \texorpdfstring{\acrNoHyperlink{\acs}{TOV}}{TOV} and \texorpdfstring{\acrNoHyperlink{\acs}{LE}}{LE} equation}
\label{subsec:4-NumSol-Sec-TOV-Exponents}
Having discussed individual solutions for the \ac{TOV} and \ac{LE} equation, this section now turns to analysing zero values of both equations.
We define a zero value as the first point in which the solution of our differential equation reaches value $0$, or more precisely where the pressure $p$ reaches zero.\footnote{Note that $p_0$ still denotes the initial value while $r_0$ and $\xi_0$ denote values where $p=0$.}
Even now the problem is not well posed since for different solving routines alternative transformations may be used.
This has to be kept in mind when comparing results. 
Apart from the previously defined exact solutions of Table~\ref{fig:3-Mass-Tbl-LE-Exact-Results} we will only use the parameter $r$ as defined by equation~\eqref{3-Mass-TOV-Eq} to display final results.
In order to solve the equations, a transformation in the form applied in section~\ref{subsec:3-Mass-Sec-LEDerivation} is used since it was hoped to increase the precision with which the zero value is determined.
This can be motivated by looking at plots for \ac{LE} solutions as in Figure~\ref{fig:3-Mass-Plt-LE-Exact-Results-Plots} and comparing them to \ac{TOV} results of Figure~\ref{fig:4-NumSol-Plt-TOVEqEasyEOS}.
With the \ac{eos} $\rho=Ap^{1/\gamma}$ and the new definition $\rho=\rho_0\theta^n$, the equations used are 
\begin{alignat}{3}
	\frac{\partial m}{\partial\xi} &= &&4\pi\kappa^3\rho\xi^2\\
	\frac{\partial\theta}{\partial\xi} &= -&&\frac{\left(1+K\rho_0^{1/n}\theta\right)\left(4\pi\xi^3\kappa^3 K\rho_0^{1+1/n}\theta^{n+1}+ m\right)}{\left((n+1)K\rho_0^{1/n}\kappa\xi^2\right)\left(1-\frac{2 m}{\kappa\xi}\right)}
	\label{eq:4-NumSol-Equ-TOV-Exponents-Transf-TOV}
\end{alignat}
where $\rho=\rho_0\theta^n$ and initial values $\partial m/\partial\xi=0$ and $\partial\theta/\partial\xi=0$ can again be calculated with L'Hospitals rule together with $\theta_0=1$ and $ m_0=0$.
When comparing equations~\eqref{eq:4-NumSol-Equ-TOV-Exponents-Transf-TOV} and~\eqref{3-Mass-TOV-Eq} one can immediately recognise the distinct terms.
When talking about zero values we stated earlier that transformations need be accounted when comparing results.
However, in this case since
\begin{equation}
	p=\frac{\rho_0^n\theta^n}{A^{1+1/n}}
	\label{eq:4-NumSol-Pressure-Theta-Relation}
\end{equation}
it is clear that zero values of $\theta$ and $p$ match under the same parameter $\xi$.
When transforming back by using $\kappa\xi=r$ as explained in section~\ref{subsec:3-Mass-Sec-LEDerivation} one should note that $\kappa$ depends on $n$ via
\begin{equation}
	4\pi\kappa^2=(n+1)K\rho_0^{1/n-1}
	\label{eq:4-NumSol-Conversion-Factor}
\end{equation}
and it will thus change values by more than constant scaling.
Figure~\ref{fig:4-NumSol-Plt-TOV-Exponents-Combo} shows zero values for solutions of the \ac{TOV} and \ac{LE} equations respectively with their dependence on $n$.
Since not only the parameters $A$ and $n$ in the \ac{eos} may alter results but also the initial value $p_0$ a family of curves was plotted.
The upper half of the plot displays a currently not explainable bump which is prominent in the \ac{TOV} $p_0=0.1$ case.
A similar but dampened behaviour can also be seen along the $p_0=1$ curve.
It is however still unclear if the \ac{TOV} solutions similarly to the \ac{LE} solutions blow up at the same combination of parameters.
In order to further analyse this property higher integration ranges for larger values of the polytropic index $n$ would be required.
This observation motivates a theorem that for every combination of parameters $A$ and $p_0$ there exists a value $n_0$ such that each solution of the TOV equation has no zero values for that particular combination of parameters $A,p_0$ if $n\geq n_0$.
The methods to prove this theorem will be discussed in the next section.\\
To obtain Figure~\ref{fig:4-NumSol-Plt-TOV-Exponents-Combo} the differential equations need to be solved multiple times for varying parameter combinations until the pressure reaches its zero value.
Numerical optimisations to speed up calculation times, and a short discussion on the methods used can be found in the appendix in section~\ref{subsec:99-App-Numerical-Optimisations}.
\begin{figure}[H]
	{\centering
	\import{pictures/4-NumericalSolutions/}{TOV-Exponents-LESubs-InitialVals-Database-PlotResults-Combo.pgf}
	}
	\caption{Zero Values of TOV and LE equation}
	\label{fig:4-NumSol-Plt-TOV-Exponents-Combo}
	In the first plot results for $p_0=8$ have been omitted for better visual clarity.
	These behave similar to the $p_0=1$ results.
\end{figure}
\end{subsection}
%
%
\end{section}
