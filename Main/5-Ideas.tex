\section{Ideas}
\begin{theorem}[Lane-Emden Finite Boundary]
	\label{Theo-Lane-EmdenFiniteBoundary}
	For every $0\leq n\leq5$, the Lane-Emden equation
	\begin{equation}
		\frac{d^2\theta}{d\xi^2}+\theta^n=0
		\label{5-Ideas-LEeq}
	\end{equation}
	with $\theta(\xi=0)=1$ has a zero value for a finite $\xi_0$. To prove this theorem, show the statements \ref{Theo-Lane-EmdenFiniteBoundary1}, \ref{Theo-Lane-EmdenFiniteBoundary2} and \ref{Theo-Lane-EmdenFiniteBoundary3}.
\end{theorem}
\begin{theorem}[Lane-Emden Finite Boundary 1]
	\label{Theo-Lane-EmdenFiniteBoundary1}
	For every $n\geq0$, the Lane-Emden equation \ref{5-Ideas-LEeq} has a solution in a $\epsilon$ Ball around $r=0$.
\end{theorem}
\begin{theorem}[Lane-Emden Finite Boundary 2]
	\label{Theo-Lane-EmdenFiniteBoundary2}
	For $n<5$, the Lane-Emden equation \ref{5-Ideas-LEeq} has no Solution defined on the total space $\R_{\geq0}$.
\end{theorem}
\begin{theorem}[Lane-Emden Finite Boundary 3]
	\label{Theo-Lane-EmdenFiniteBoundary3}
	Let $n<5$ and $\theta:[0,x)\rightarrow\R$ be a solution of \ref{5-Ideas-LEeq}. If $\theta(x)>0$, then $\theta$ can be extended with Theorem \ref{Theo-Lane-EmdenFiniteBoundary1} to $[0,x+\epsilon)$ where $\epsilon$ follows a growth condition $\epsilon(\xi)=...$ such that if $\theta$ would have no zero point, there would be a solution on total $\R_{\geq0}$ and thus a contradiction with \ref{Theo-Lane-EmdenFiniteBoundary2}.
\end{theorem}
\begin{theorem}[Lane-Emden Finite Boundary 4]
	The Lane Emden equation has a zero value if the exponent function $n(\xi)$ fullfills the growth condition XYZ.
\end{theorem}

\begin{theorem}[TOV Exact Solution]
	The TOV equation \ref{4-NumSol-TOVEqBasic1} with a polytropic EOS $\rho=Ap^{1/\gamma}$ has a well defined limiting case where $A\rightarrow0$ with $m=0$ and
	\begin{equation}
		p = \frac{1}{r^2+1} \text{ mod } 4\pi.
	\end{equation}
\end{theorem}
\begin{theorem}[TOV Finite Boundary]
	For each $A>0$, the TOV equation \ref{4-NumSol-TOVEqBasic1} has a exponent $n>0$ for which a solution $p$ does not have any zero points.
\end{theorem}
\begin{theorem}[TOV Finite Boundary 1]
	If $p_1$ and $p_2$ are two solutions of the TOV equation (for equal $A$) with $p_i$ corresponding to $n_i$ exponents, then if $n_1<n_2$, the zero point (if it exists) of $p_1$ is smaller than that of $p_2$ (if it exists or is $\infty$).
\end{theorem}
\begin{theorem}[TOV Finite Boundary 2]
	a
\end{theorem}




