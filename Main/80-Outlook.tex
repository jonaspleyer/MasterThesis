\section{Conclusion}
\label{sec:80-Outlook}
This thesis has dealt with the question to develop methods and describe zero values of the \ac{TOV} and \ac{LE} equation.
We demonstrated how to implicitly derive a fully special relativistic \ac{eos} in the example of a non-interacting gas (section~\ref{sec:2-Thermo}).
It was derived by using an explicit result for the partition function $\mathcal{Z}$ in equation~\ref{2-IntEner-PartFunc}, calculating the internal energy $\mathcal{U}$ and pressure $p$.
Afterwards an implicit relation between the energy density $\mathcal{U}/V$ and the pressure $p$ was found.
These results are also applicable in a wider context when generalising to special relativistic treatments of thermodynamics.
Future work could be carried forward in the context of interacting gases or by introducing quantum effects and varying constituents.\\
Further, the derived \ac{eos} and a standard polytropic \ac{eos} $\rho=Ap^{1/\gamma}$ were applied in numerical solutions of the \ac{TOV} equation.
Exact results in the non-relativistic \ac{LE} case (section~\ref{subsec:5-zeroes-le-equation}) show that this equation has finite zero values in the range $1<n<5$ of the polytropic index.
Numerical results of the relativistic \ac{TOV} case with a polytropic \ac{eos} demonstrate a similar behaviour.
We assess the question if hypothesis~\ref{theo:5-Zeroes-TOV-Zero-Value-Hypothesis} is provable by a similar procedure as in the \ac{LE} case.
We observe that its proof~\cite{quittnerSuperlinearParabolicProblems2007} (shortened version see section~\ref{subsec:99-App-NoGlobalLE}) heavily relies on pieces such as Bochner's identity or gradient estimates.
Although some of these statements directly survive when going to a curved space, it is unclear how the \ac{TOV} equation could be reformulated in terms of the Laplace operator on a spacelike hypersurface, in order to apply said arguments.
It's an affair that can be attributed to the non-linear first order nature of the Einstein equations~\ref{eq:01-Intr-Einstein-Equ}.\\
Another ansatz is to compare solutions to functions with known zero values.
Statements about global solutions such as Theorem XIII in~\cite[p. 99]{walterOrdinaryDifferentialEquations1998} could prove useful in this scenario but would require good approximations of the defect $(m,p)'-U(m,p)$ (see equations~\ref{eq:5-Zeroes-TOV-Operator-Def-1},\ref{eq:5-Zeroes-TOV-Operator-Def-2}) of the \ac{TOV} equation.\\
An additional question raised is how the behaviour of zero values changes if rotating or charged stars are being considered.
A stellar object in the Kerr metric~\cite{kerrGravitationalFieldSpinning1963} would be expected to yield smaller zero values compared to \ac{TOV} results due to a in general larger attractive force.
Since spherical symmetry in this case is not a valid assumption anymore, a new derivation is required which could however readily use the developed results and methods.
We expect similar behaviour in the charged case.\\
In the pursuit of classifying stellar model properties this thesis has improved on previous knowledge and left new and interesting questions unanswered.
The developed numerical methods are readily available and easy to use while the obtained results provide valuable background information to any theoretical astrophysicist.