\section{Outlook}
\label{sec:80-Outlook}
This thesis has demonstrated how to implicitly derive a fully special relativistic \ac{eos} in the example of a non-interacting gas.
It has also applied said relation in the \ac{TOV} equation and shown numerical results.
Furthermore, exact results in the non-relativistic \ac{LE} equation show that zero-values of this equation are finite in the range $1<n<5$ of the polytropic index.
Numerical results of the relativistic \ac{TOV} case with a polytropic \ac{eos} demonstrate a similar behaviour.
We assess the question if hypothesis~\ref{theo:5-Zeroes-TOV-Zero-Value-Hypothesis} is provable by a similar procedure as in the \ac{LE} case.
First observe that its proof which can be found in~\cite{quittnerSuperlinearParabolicProblems2007} or in a shortened version in section~\ref{subsec:99-App-NoGlobalLE} heavily relies on pieces such as Bochner's identity or estimating gradients which would hold in curved spaces.
However, it is unclear how the \ac{TOV} equation could be reformulated in terms of the Laplace operator.
An affair that can be attributed to the non-linear first order nature of the Einstein equations~\ref{eq:01-Intr-Einstein-Equ}.
% TODO Write something about rotating and charged stars
A possible approach to tackle this problem would be in comparing solutions to functions with known zero-values.
Statements about global solutions such as Theorem XIII in~\cite[p. 99]{walterOrdinaryDifferentialEquations1998} could prove useful in this scenario but would require good approximations of the defect $v'-f(x,v)$ of the \ac{TOV} equation.\\
Another question raised is how the behaviour of zero values changed if rotating or charged stars were considered.
A stellar object in the Kerr metric~\cite{kerrGravitationalFieldSpinning1963} would be expected to yield smaller zero values compared to \ac{TOV} results due to a in general larger attractive force.



In the pursuit of classifying stellar model properties