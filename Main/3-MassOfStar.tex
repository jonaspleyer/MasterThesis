\section{Calculating the Mass of a Star with an \texorpdfstring{\acrNoHyperlink{\acs}{eos}}{EoS}}
\label{sec:3-Mass}
\subsection{Deriving the \texorpdfstring{\acrNoHyperlink{\acs}{TOV}}{TOV}-Equation}
\label{subsec:3-Mass-Sec-TOVDerivation}
This chapter derives the \ac{TOV} equation which was introduced earlier in section~\ref{sec:01-Introduction}.
Information about \ac{gr} can be found in~\cite{choquet-bruhatAnalysisManifoldsPhysics2000, choquet-bruhatGeneralRelativityEinstein2009, choquet-bruhatIntroductionGeneralRelativity2015, waldGeneralRelativity1984}.
We follow the derivation of Wald.
We consider a spherical-symmetric static Lorentz-Manifold $(V,g)$ with charts such that the metric $g$ can be written as
\begin{equation}
	g=-\e^\nu \diff t^2+\e^\lambda \diff r^2 + r^2(\diff\theta^2+\sin^2\theta \diff\phi^2).
\end{equation}
The stress-energy tensor of an ideal fluid with density $\rho$ and pressure $p$ is given by
\begin{equation}
	T_{\mu\nu}=\rho u_\mu u_\nu + p(g_{\mu\nu}+u_\mu u_\nu)
\end{equation}
where $u$ is the 4-velocity of the fluid.
In the rest frame where $u^\mu=(-\e^{-\nu/2},0,0,0)$, this equation simplifies to
\begin{equation}
	T^\mu_\nu=\text{diag}(-\rho,p,p,p)
\end{equation}
The Christoffel symbols for this metric are
\begin{align}
	\Gamma_{\mu\nu}^0 &= \begin{bmatrix}
	                     	0 & \nu'/2 & & \\
	                     	\nu'/2 & 0 & & \\
	                     	& & 0 &\\
	                     	& & & 0
	                     \end{bmatrix}, \hspace{0.3cm}
	\Gamma_{\mu\nu}^1 = \begin{bmatrix}
	                     	\nu'\e^{\nu-\lambda}/2 & & &\\
	                     	& \lambda'/2 & & \\
	                     	& & -r\e^{-\lambda} & \\
	                     	& & & -r\sin^2\theta\e^{-\lambda}
	                     \end{bmatrix}\\
	\Gamma_{\mu\nu}^2 &= \begin{bmatrix}
	                     	0 & & &\\
	                     	& 0 & 1/r &\\
	                     	& 1/r & 0 &\\
	                     	& & & -\sin\theta\cos\theta
	                     \end{bmatrix}, \hspace{0.3cm}
	\Gamma_{\mu\nu}^3 = \begin{bmatrix}
	                     	& 0 & 0 &\\
	                     	& 0 & 0 & 1/r\\
	                     	& 0 & 0 & \cos\theta/\sin\theta\\
	                     	& 1/r & \cos\theta/\sin\theta & 0
						\end{bmatrix}
\end{align}
From these, the non-zero components of the Ricci-Tensor can be calculated 
\begin{align}
	R_{11} &= \frac{1}{4r}\e^{-\lambda}\left[\left(2r\nu''+r\nu'^2\right) + \left(4-r\lambda'\right)\nu'\right]\\
	R_{22} &= -\frac{1}{4r}\e^{-\lambda}\left[\left(2r\nu''\right)+r\nu'^2-r\lambda'\nu'-4\lambda'\right]\\
	R_{33} &= -\frac{1}{2r^2}\e^{-\lambda}\left(r\nu'-r\lambda'-2\e^\lambda+2\right)\\
	R_{44} &= R_{33}
\end{align}
and with $R_{\mu\nu}-g_{\mu\nu}R/2=G_{\mu\nu}=8\pi T_{\mu\nu}$ we ultimately receive the following field equations.
\begin{align}
	-8\pi T_0^0 = 8\pi\rho &= \frac{\lambda'\e^{-\lambda}}{r}+\frac{1-\e^{-\lambda}}{r^2}\label{eq:3-Mass-Equ-EinstEqu-1}\\
	8\pi T_1^1 = 8\pi p &= \nu'\frac{\e^{-\lambda}}{r} - \frac{1-\e^{-\lambda}}{r^2}\label{eq:3-Mass-Equ-EinstEqu-2}\\
	8\pi T_2^2 = 8\pi p &= \frac{\e^{-\lambda}}{2}\left[\nu''+\left(\frac{\nu'}{2} + \frac{1}{r}\right)\left(\nu'-\lambda'\right) \right]\label{eq:3-Mass-Equ-EinstEqu-3}
\end{align}
Since $R_4^4=R_3^3$, we omitted the last equation.
From equation \eqref{eq:3-Mass-Equ-EinstEqu-1} we infer the relation.
\begin{equation}
	\e^{-\lambda} = 1 - \frac{2}{r}\int\limits_0^r 4\pi\rho(r') r'^2\diff r' = 1 - \frac{2m(r)}{r}.
	\label{eq:3-Mass-Equ-MassDepOnr}
\end{equation}
The metric needs to be defined at every point in space and thus can not have any additional integration constant in equation~\eqref{eq:3-Mass-Equ-MassDepOnr}, since otherwise we would obtain a term $a/r$ which is not defined for $r\rightarrow0$.\\
The property $m(r)$ can be recognised as the Newtonian mass of the star (which is different to the proper mass~\cite{waldGeneralRelativity1984}).
Since $\e^{-\lambda}>0$, we immediately see that $m(r)<r/2$.\\
In addition to the Field equations \eqref{eq:3-Mass-Equ-EinstEqu-1} through~\eqref{eq:3-Mass-Equ-EinstEqu-3} the divergence of the Stress-Energy Tensor yields more information
\begin{equation}
	\nabla_\mu T^{\mu\nu}=0.
\end{equation}
The following explicit calculation, which again assumes spherical symmetry, shows how to obtain this additional restriction on the pressure and density.
\begin{align}
	\nabla_\mu T^\mu_\nu 	&= \partial_\mu T^\mu_1 + \Gamma^\mu_{\mu\sigma}T^\sigma_\nu-\Gamma^\sigma_{\mu\nu}T^\mu_\sigma\\
	\nabla_\mu T^\mu_1		&= \frac{\partial p}{\partial r} + p\left(\Gamma^0_{01}+\Gamma^1_{11}+\Gamma^2_{21}+\Gamma^3_{31} \right) - \Gamma^\sigma_{\mu 1}T^\mu_\sigma\\
							&= \frac{\partial p}{\partial r} + p\left(\frac{\nu'+\lambda'}{2} + \frac{2}{r}\right) + \rho\frac{\nu'}{2} - p\frac{\lambda'}{2} - p\frac{2}{r}\\
	\frac{\partial p}{\partial r} &= -\frac{p+\rho}{2}\nu'
\end{align}
Together with equation~\eqref{eq:3-Mass-Equ-EinstEqu-2} and the definition~\eqref{eq:3-Mass-Equ-MassDepOnr}, we can write
\begin{align}
	\frac{\partial p}{\partial r} 	&= -\frac{p+\rho}{2}\left(\frac{8\pi pr + \frac{1-\e^{-\lambda}}{r}}{\e^{-\lambda}} \right)\\
									&= -\frac{p+\rho}{2r}\left(\frac{8\pi pr+ \frac{2m}{r^2}}{1-\frac{2m}{r}} \right)\\
									&= -\frac{m\rho}{r^2}\left(1+\frac{p}{\rho}\right)\left(\frac{4\pi r^3 p}{m}+1\right)\left(1-\frac{2m}{r}\right)^{-1}\\
	\frac{\partial p}{\partial r} 	&= -\frac{Gm\rho}{r^2}\left(1+\frac{p}{\rho c^2}\right)\left(\frac{4\pi r^3 p}{mc^2}+1\right)\left(1-\frac{2Gm}{rc^2}\right)^{-1}
	\label{eq:3-Mass-Equ-TOV-Eq-1}
\end{align}
where in the last step the constants $c=G=1$ were put back in.
Equation~\eqref{eq:3-Mass-Equ-TOV-Eq-1} together with~\eqref{eq:3-Mass-Equ-MassDepOnr} yields the \ac{TOV} differential equations
\begin{alignat}{3}
	\frac{\partial m}{\partial r} &= &&4\pi\rho(r)r^2\\
	\frac{\partial p}{\partial r} &= -&&\frac{Gm\rho}{r^2}\left(1+\frac{p}{\rho c^2}\right)\left(\frac{4\pi r^3 p}{mc^2}+1\right)\left(1-\frac{2Gm}{rc^2}\right)^{-1}
	\label{3-Mass-TOV-Eq}
\end{alignat}
%
%
%
\pagebreak
\subsection{Newtonian Limit}
\label{subsec:3-Mass-Sec-LEDerivation}
This section follows~\cite{weissteinLaneEmdenDifferentialEquation2020} and~\cite[89\psqq]{chandrasekharChandrasekharAnIntroductionStudy1958}.
Together with a polytropic \ac{eos} $p=K\rho^{1+1/n}$ and the definition $\rho=\lambda\theta^n$, we expect to obtain the Newtonian behavior in the non-relativistic limit in the form of the \ac{LE} equation
\begin{equation}
	\frac{K(n+1)\lambda^{1/n-1}}{4\pi}\Delta\theta+\theta^n=0.
\end{equation}
The usual non-relativistic limit is obtained from a Taylor expansion of equation \eqref{3-Mass-TOV-Eq} around $1/c^2$ in lowest order.
The resulting equation then reads
\begin{equation}
	\frac{\partial p}{\partial r} = -\frac{Gm\rho}{r^2} + \mathcal{O}\left(\frac{1}{c^2}\right).
	\label{eq:3-Mass-Equ-TOVNonRel-Limit}
\end{equation}
One could be tempted to make the assumption that $\frac{\partial p_{\text{TOV}}}{dr} \leq \frac{\partial p_{\text{LE}}}{dr}.$ However this equation fails since the mass given in equation \eqref{3-Mass-TOV-Eq} is not the same as the one given in the \ac{LE} equation. Using the previous relations for $\rho$ and $p$ and again setting $G=c=1$ , we can calculate
\begin{equation}
	\frac{\partial p}{\partial r} = \frac{\partial}{\partial r}\left(K\rho^{1+1/n}\right)= K\lambda^{1+1/n}(n+1)\theta^n\frac{\partial\theta}{\partial r} = -\frac{m\lambda\theta^n}{r^2}
\end{equation}
by using the definition of our polytropic \ac{eos} and equation \eqref{eq:3-Mass-Equ-TOVNonRel-Limit}.
Rearranging and taking the derivative of this equation and using $\partial m/\partial r = 4\pi\rho r^2$, we obtain
\begin{equation}
	- \frac{\partial m}{\partial r} = K\lambda^{1/n}(n+1)\frac{\partial}{\partial r}\left(r^2\frac{\partial\theta}{\partial r}\right) = -4\pi r^2\lambda\theta^n
\end{equation}
% \begin{equation}
% 	\frac{K\lambda^{1/n-1}(n+1)}{4\pi}\Delta\theta+\theta^n=0.
% \end{equation}
Upon redefining $\xi=r/\kappa$ where $4\pi\kappa^2=(n+1)K\lambda^{1/n-1}$, one can obtain the mathematically cleaner looking equation
\begin{equation}
	\frac{1}{\xi^2}\frac{\partial}{\partial\xi}\left(\xi^2\frac{\partial\theta}{\partial\xi}\right) + \theta^n=0
	\label{eq:3-Mass-Equ-Lane-Emden-Eq}
\end{equation}
Exact solutions are known for the cases $n=0,1,5$.
The derivation can be found in appendix~\ref{99-App-A-Exact-LE-Solutions}.
Figure~\ref{fig:3-Mass-Plt-LE-Exact-Results-Plots} and Table~\ref{fig:3-Mass-Tbl-LE-Exact-Results} summarise them.
Together with equation~\eqref{eq:3-Mass-Equ-Lane-Emden-Eq}, we can already suspect that for values $n\geq5$, the equation does not yield solutions with a zero value.\\
% \begin{figure}[H]
% 	\centering
% 	\begin{tabular}{p{\dimexpr 8cm}p{\dimexpr \linewidth-4\tabcolsep-8cm}}
% 		\import{pictures/3-MassOfStar/}{LE-SingleSolve.pgf}
% 		&
% 		\renewcommand{\arraystretch}{1.2}
% 		\begin{tabular}[b]{@{}lcccc@{}}
% 			\toprule
% 			$n$ & \phantom{abcd} & \ac{LE} Solution & \phantom{abcd} & $\xi_0$\\
% 			\cmidrule{1-5}
% 			$n=0$ && $\displaystyle 1-\frac{1}{6}\xi^2$ && $\sqrt{6}$\\[3ex]
% 			$n=1$ && $\displaystyle \frac{\sin(\xi)}{\xi}$ && $\pi$\\[3ex]
% 			$n=5$ && $\displaystyle \frac{1}{\sqrt{1+\frac{1}{3}\xi^2}}$ && $\infty$\\[2ex]
% 			\bottomrule
% 		\end{tabular}\\
% 		\caption[Graph of exact Lane Emden Solutions]{Graph of exact \ac{LE} solutions.}
% 		\label{3-Mass-Plt-LE-Exact-Results-Plots}
% 		&
% 		\captionof{table}[Lane Emden exact Solutions]{\ac{LE} exact Solutions and their zero value.}
% 		\label{3-Mass-Plt-LE-Exact-Results}
% 	\end{tabular}
% \end{figure}
\noindent
\begin{minipage}{0.5\textwidth}
	\centering
	\import{pictures/3-MassOfStar/}{LE-SingleSolve.pgf}
\end{minipage}\hfill%
\begin{minipage}{0.45\textwidth}
	\renewcommand{\arraystretch}{1.2}
	\begin{tabular}[b]{@{}lcccc@{}}
		\toprule
		$n$ & \phantom{ab} & \ac{LE} Solution & \phantom{ab} & $\xi_0$\\
		\cmidrule{1-5}
		$n=0$ && $\displaystyle 1-\frac{1}{6}\xi^2$ && $\sqrt{6}$\\[3ex]
		$n=1$ && $\displaystyle \frac{\sin(\xi)}{\xi}$ && $\pi$\\[3ex]
		$n=5$ && $\displaystyle \frac{1}{\sqrt{1+\frac{1}{3}\xi^2}}$ && $\infty$\\[2ex]
		\bottomrule
	\end{tabular}
\end{minipage}
\begin{minipage}[t]{0.5\textwidth}
	\begin{figure}[H]
		\caption[Graph of exact Lane Emden Solutions]{Graph of exact \ac{LE} solutions.}
		\label{fig:3-Mass-Plt-LE-Exact-Results-Plots}
	\end{figure}
\end{minipage}\hfill%
\begin{minipage}[t]{0.45\textwidth}
	\begin{figure}[H]
		\captionof{table}[Lane Emden exact Solutions]{\ac{LE} exact Solutions and their zero value.}
		\label{fig:3-Mass-Tbl-LE-Exact-Results}
	\end{figure}
\end{minipage}
%
%
%
%
\subsection{Mass Bounds}
\label{subsec:3-Mass-Mass-Bounds}
%The aim of this section is to derive an upper limit for the mass of a spherically symmetric stellar object which obeys an local \ac{eos} $f(p,\rho)=0$. 
We will first follow the approach given in \cite{waldGeneralRelativity1984}. 
The first assumptions will be $d\rho/dr<0$ and $\rho\geq0$. 
Also we again consider a compact star, meaning $\rho(r)=0$ for all $r>R$.
% While we talk about the derivative of $\rho$ it may not be differentiable at $r=R$.
% However differentiability of the metric demands at least continuity at every point. 
%
%
% In total we assume the following Definition.
% \begin{definition}[Star]
% 	A star is a compact subset $K\subseteq V$ that is diffeomorph (in the sense of a submanifold of $\R^3$) to a closed Ball $B_1(0)$. The center of the star is defined by the equation
% 	\begin{equation}
% 		x_{\text{center}}=\frac{1}{vol(K)}\int\limits_K x \omega
% 	\end{equation}
% 	where $\omega=*(1)$ is the uniquely\footnote{Since $B_1(0)$ is orientable.} defined volume form on $K$. If not mentioned otherwise, we assume to be $x_{center}=0$. If additionally $K$ is isometric to $B_1(0)$ then we call $K$ spherically symmetric.
% \end{definition}
% \begin{definition}[Density Function and Mass]\label{3-Mass-DefDensFuncAndMass}
% 	A density function $\rho:\R^3\rightarrow\R_{\geq0}$ of a compact star $K$ is a function that is differentiable on the subset $\text{interior}(K)$ with $\partial_{x_i}\rho(x)\leq0$ and has $supp(\rho)=K$. If additionally, $K$ is spherically symmetric and $\rho$ is to be invariant under rotations, meaning $\rho(\Lambda x)=\rho(x)$ for every $\Lambda\in O(3)$, then $\rho$ is called spherically symmetric and factorises through the function $x\mapsto|x|$ and is thus simply written as\footnote{With abuse of Notation} $\rho:\R_{\geq0}\rightarrow\R_{\geq0},r\mapsto\rho(r)$. If not mentioned differently a density function of a spherically symmetric star is always chosen to be spherically symmetric. The mass $M\geq0$ of a star is given by 
% 	\begin{equation}
% 		M = \int\limits_K\rho(x)\omega
% 	\end{equation}
% 	where $\omega$ is again the uniquely defined volume form on $K$.
% \end{definition}
We first state a useful Lemma for the proof of our next theorem.
\begin{lemma}
	Given $\rho\in C^1(\R_{\geq0})$ is monotonously decreasing, the function $\rho_{av}=m/r^3$ has negative slope.
\end{lemma}
\begin{proof}
	Taking the derivative of $\rho_{av}$ and applying equation~\eqref{eq:3-Mass-Equ-MassDepOnr}, we obtain
	\begin{equation}
		\frac{\partial}{\partial r}\left(\frac{m}{r^3}\right) = -3\frac{m}{r^4} + \frac{4\pi\rho}{r}.
	\end{equation}
	By equation~\eqref{3-Mass-TOV-Eq} we know that $p$ has negative slope and since $\rho$ is monotonously increasing, we have
	\begin{equation}
		m = \int\limits_0^r 4\pi\rho r^2 \diff r \geq \rho\int\limits_0^r4\pi r^2 \diff r= \frac{4\pi r^3\rho}{3}
	\end{equation}
	which shows that $\frac{4\pi\rho r^3}{3}\leq m$ and completes the proof.
\end{proof}\noindent
\begin{theorem}[Mass Bound]
	The Mass of a spherically symmetric star is bound from above by
	\begin{equation}
		M < \frac{4}{9}R.
	\end{equation}
\end{theorem}
\begin{proof}
	In our attempt to obtain an upper limit for the mass of a spherically symmetric star, we start by taking the difference of equation \eqref{eq:3-Mass-Equ-EinstEqu-2} and \eqref{eq:3-Mass-Equ-EinstEqu-3}, we obtain
	\begin{align}
		0 &= \nu'\frac{\e^{-\lambda}}{r} - \frac{1-\e^{-\lambda}}{r^2} - \frac{\e^{-\lambda}}{2}\left[\nu''+\left(\frac{\nu'}{2} + \frac{1}{r}\right)\left(\nu'-\lambda'\right) \right]\\
		&= - \frac{2m(r)}{r^3} + \frac{\lambda'\e^{-\lambda}}{2r} -  \frac{\e^{-\lambda}}{2}\left[\nu''+ \frac{\nu'^2}{2} - \frac{\nu'}{r} -\frac{\lambda'\nu'}{2}\right]\\
		&= r\frac{\partial}{\partial r}\left(\frac{m(r)}{r^3}\right) - \frac{\e^{-\lambda}}{2}\left[\nu''+ \frac{\nu'^2}{2} - \frac{\nu'}{r} -\frac{\lambda'\nu'}{2}\right]\\
		0 &= \frac{\partial}{\partial r}\left(\frac{m(r)}{r^3}\right) - \frac{\e^{-\lambda}}{2}\left[\frac{\nu''}{r}+ \frac{\nu'^2}{2r} - \frac{\nu'}{r^2} -\frac{\lambda'\nu'}{2r}\right]\\
		&= \frac{\partial}{\partial r}\left(\frac{m(r)}{r^3}\right) - \frac{1}{2}\e^{-\frac{\lambda+\nu}{2}}\frac{\partial}{\partial r}\left[\frac{1}{r}\nu'\e^{\frac{\nu-\lambda}{2}}\right].
	\end{align}
	Since $\partial_r\rho\leq0$, also the average density $m(r)/r^3$ decreases with $r$.
	Thus we obtain
	\begin{equation}
		\frac{\partial}{\partial r}\left[\frac{1}{r}\nu'\exp\left(\frac{\nu-\lambda}{2}\right) \right] \leq 0.
	\end{equation}
	We integrate this equation from $R$ to $r<R$
	\begin{equation}
		\frac{\nu'}{r}\exp\left(\frac{\nu-\lambda}{2}\right)\geq\frac{2\nu'(R)}{R}\e^{-\frac{1}{2}\lambda(R)}\left.\frac{\partial}{\partial r}\e^{\frac{\nu}{2}}\right|_R
	\end{equation}
	and use the Schwarzschild solution at $r=R$ for $\e^\lambda$ and $\e^\nu$. 
	This is justified since we assumed $\rho(r)=0$ for $r>R$ and thus we need to recover the vacuum solution for a spherically symmetric object which is given by the Schwarzschild solution. 
	By continuity of the metric on every point of space, we can match 
	\begin{equation}
		\left.\e^{-\lambda(r)}\right|_R=\left[1-\frac{2M}{r}\right]_R=\left.\e^{\nu(r)}\right|_R
		\label{eq:3-Mass-Equ-MassLimitApprox1}
	\end{equation}
	and with the explicit solution for $\e^{-\lambda}$, we obtain
	\begin{equation}
		\left.\frac{2m(r)}{r}\right|_R = \frac{2M}{R}.
	\end{equation}
	When plugging this into equation~\eqref{eq:3-Mass-Equ-MassLimitApprox1}, the result is
	\begin{equation}
		\frac{\nu'}{2r}\exp\left(\frac{\nu-\lambda}{2}\right)\geq\frac{(1-2M/R)^{1/2}}{R}\left.\frac{\partial}{\partial r}\left(1-\frac{2M}{r}\right)^{1/2}\right|_{r=R} = \frac{M}{R^3}.
	\end{equation}
	Now we multiply by $r\exp(\lambda/2)$ and use the explicit solution for $\e^\lambda$
	\begin{equation}
		\frac{\partial}{\partial r}\left(\e^{\frac{\nu}{2}}\right) \geq \frac{M}{R^3}r\e^\frac{\lambda}{2} = \frac{M}{R^3}\left(r-2m(r)\right)
	\end{equation}
	and integrate again this time from $0$ to $R$
	\begin{equation}
		\e^{\nu(0)/2}\leq\left(1-\frac{2M}{R}\right)^{1/2}-\frac{M}{R^3}\int\limits_0^R\left[1-\frac{2m(r)}{r} \right]^{-1/2}r\diff r.
		\label{eq:3-Mass-Equ-MassLimitNuApprox}
	\end{equation}
	As we have already noted, the average density $m(r)/r^3$ decreases, meaning explicitly $m(r)/r^3\geq M/R^3$ and thus the integral with the previous equation can be written as
	\begin{equation}
		\e^{\nu(0)/2}\leq\left(1-\frac{2M}{R}\right)^{1/2}+\frac{1}{2}\left.\left[1-\frac{2Mr^2}{R^3} \right]^{1/2}\right|^{r=R}_{r=0} = \frac{3}{2}\left(1-\frac{2M}{R}\right)^{1/2} - \frac{1}{2}.
	\end{equation}
	The simple fact that $\e^{\nu(0)/2}>0$ then implies
	\begin{equation}
		\left(1-\frac{2M}{R}\right)^{1/2} > \frac{1}{3}
	\end{equation}
	which is equivalent to
	\begin{equation}
		M< \frac{4R}{9}.
	\end{equation}
\end{proof}\noindent
This shows that the mass of star has an upper limit under the assumptions $\rho\geq0$, $\partial_r\rho\leq0$ and $\rho(R)=0$ for some $R\geq0$.
In particular by restraining the total radius of the stellar object one can immediately find mass limits.
% The case in which $M=4R/9$ would be achieved by using a constant density of 
% \begin{equation}
% 	\rho = \frac{1}{4\pi}\frac{M}{R^3}
% \end{equation}
% from which the mass $m(r)$ can be obtained
% \begin{equation}
% 	m(r) = \frac{r^3}{R^3}M.
% \end{equation}
% However this case is actually forbidden as was just shown. 
% Physically, the assumption that $\rho(R)=0$ defines a radius for the stellar object since it limits the physical dimension. In general stellar objects need not to fulfill this condition. Physically it is however necessary to have $p(R)=0$ while the density may have discontinuities at this point.