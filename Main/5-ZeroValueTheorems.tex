\section{To a General Zero Value Theorem}
\subsection{\acrNoHyperlink{\acs}{LE} Equation}
Motivated by the preceding sections, the aim now is to prove the following theorem.
\begin{theorem}[Lane-Emden Finite Boundary]
	\label{5-Zeroes-The-Lane-EmdenFiniteBoundary}
	For every $0\leq n<5$, the Lane-Emden equation
	\begin{equation}
		\frac{1}{\xi^2}\frac{d}{d\xi}\left(\xi^2\frac{d\theta}{d\xi}\right)+\theta^n=0
		\label{5-Zeroes-Equ-LEeq}
	\end{equation}
	with $\theta_0=1$ has a zero value for a finite $\xi_0$. To prove this theorem, show the following statements.
\end{theorem}\noindent
The procedure for proving existance is similar for both equations.
We want to use the Banach fixed-point theorem to show that over a suitably chosen space of functions we obtain a unique solution to the differential equation.
In order to tackle these problems we first state some aiding lemmata.
\begin{lemma}
	\label{5-Zeroes-Lem-Integral-Contraction}
	Let $f:I\times U\rightarrow V$ with $I=[a,b]$ and $U\subseteq\R^n,V\subseteq\R^m$ be continuous and $H:K\rightarrow K$ with $K\subseteq C^0(I\times U,V)$ be Lipschitz.
	Then the mapping $T:K\rightarrow K$ which is defined by
	\begin{equation}
		T(f)(t,x)=\int\limits_a^tH(f(s,x))ds
	\end{equation}
	is a contraction when choosing $t\in[a,a+\epsilon]$ with $\epsilon>0$ small enough.
\end{lemma}
\begin{proof}
	Since $f$ is Lipschitz with constant $L$ we can write
	\begin{align}
		||T(f)-T(g)||_\infty 	&\leq \int\limits_a^t||H(f(t,x))-H(g(t,x))||_\infty dt\\
								&\leq \int\limits_a^tL||f-g||_\infty\\
								&=(t-a)L||f-g||_\infty.
	\end{align}
	Now it is clear that we can choose $\epsilon>0$ small enough such that the mapping is a contraction for $t\in[a,a+\epsilon]$.
\end{proof}
\begin{lemma}
	\label{5-Zeroes-Lem-Lipschitz-Continuity-Differentiable}
	Let $f:\overline{U}\rightarrow \overline{V}$ with $U\subseteq\R^n,V\subseteq\R^m$ open subsets be be continuously partially differentiable and every partial derivative well defined on $\overline{U}$.
	Then $f$ is Lipschitz on $\overline{U}$.
\end{lemma}
\begin{proof}
	With the mean value theorem of differentiation we know that
	\begin{equation}
		\frac{|f(x)-f(y)|}{|x-y|}\leq \mathrm{sup}_{x\in\overline{U}}\left(\partial_{e_i}f(x)\right)
	\end{equation}
	where $e_i$ is the unit vector pointing in $x-y$ direction.
	The right hand side of this equation is by definition a continuous function of $e_i$ and the space of unit vectors in $\R^n$ is $B_1(0)$ and thus compact.
	This shows that for every possible combination $x,y$ we can take
	\begin{equation}
		\frac{|f(x)-f(y)|}{|x-y|}\leq \sup_{e_i\in B_1(0)}\sup_{x\in\overline{U}}\left(\partial_{e_i}f(x)\right):=L
	\end{equation}
	as the Lipschitz constant.
\end{proof}
\begin{lemma}[LE Local Existence]
	\label{5-Zeroes-Lem-Lane-EmdenFiniteBoundary1}
	For every $n\geq0$, the Lane-Emden equation \ref{5-Zeroes-Equ-LEeq} with initial values defined as above has a unique solution in an $\epsilon$ Ball around $\xi=0$.
\end{lemma}
\begin{proof}
	This proof can be generalised to $l\geq2$ dimensions. 
	Following \cite{quittnerSuperlinearParabolicProblems2007a} we see that the LE differential equation can be written as
	\begin{equation}
		\frac{d}{d\xi}\left(\xi^{l-1}\frac{d\theta}{d\xi}\right) + \xi^{l-1}\theta^n = 0.
	\end{equation}
	Integrating once and rearranging leads us to
	\begin{equation}
		\frac{d\theta}{d\xi} = \left.\frac{d\theta}{d\xi}\right|_0 - \int\limits_0^\xi\left(\frac{t}{\xi}\right)^{l-1}\theta^ndt.
		\label{5-Zeroes-Equ-LEOperator-Derivative}
	\end{equation}
	By choosing the correct initial value\footnote{One has to choose this value in order to obtain the correct limiting case when using the non-relativistic limit of the \ac{TOV} equation. See also section \ref{3-Mass-Sec-LEDerivation}} of $d\theta/d\xi=0$ we see that positive solutions to the LE equation need to have negative slope.
	This equation also shows that for $\theta_0>0$, the solution $\theta$ will be positive in a $\epsilon$ region around $\xi=0$.
	We integrate both sides again and end up with
	\begin{equation}
		\theta = \theta_0 - \int\limits_0^\xi\int\limits_0^s\left(\frac{t}{s}\right)^{l-1}\theta^ndtds.
		\label{5-Zeroes-Equ-LEOperator}
	\end{equation}
	We define $C^*$ as the space of all real continuous functions $f\geq0$ on $[0,\epsilon]$ that share the same initial value $\theta_0$ and are monotonously decreasing.
	Additionally with equation \eqref{5-Zeroes-Equ-LEOperator-Derivative} we can restrict ourselves to functions with
	\begin{equation}
		\frac{d\theta}{d\xi} \geq -\theta_0^n\frac{\xi}{m}.
		\label{5-Zeroes-Equ-LEOperator-Restrict-Slope}
	\end{equation}
	With the norm $||.||_\infty$ this space is complete.
	Inspecting the operator $T:C^*\rightarrow C^*$ defined by
% 	\footnote{The  index $1$ refers to the $C^1$ norm given by $||f||_2=||f||_\infty+||f'||_\infty+||f''||_\infty$ on the yet to be defined space.}
	equation \eqref{5-Zeroes-Equ-LEOperator} we can apply the Banach fixed-point theorem \cite{banachOperationsDansEnsembles1922} in combination with lemma \ref{5-Zeroes-Lem-Integral-Contraction}.
% 	Without loss of generality, we can assume $U\neq V$. Then
	\begin{align}
		||T(U)-T(V)||_\infty	&\leq	\left|\left|U^n-V^n\right|\right|_\infty\int\limits_0^\xi\int\limits_0^s \left(\frac{t}{s}\right)^{l-1}dtds\\
								&=		\left|\left|U^n-V^n\right|\right|_\infty\frac{\xi^2}{2l}
	\end{align}
	Let $n\geq1$. 
	Since the function $x\mapsto x^n$ is differentiable for $n\geq1$ with lemma \ref{5-Zeroes-Lem-Lipschitz-Continuity-Differentiable} it is also Lipschitz with constant $L$ on the compact interval $I=[0,\theta_0]$.
	This means in particular that we have
	\begin{equation}
		||T(U)-T(V)||_\infty	\leq	\left|\left|U-V\right|\right|_\infty\frac{L\xi^2}{2l}
	\end{equation}
	Thus when choosing $\epsilon<\sqrt{2m/L}$ we have a contraction and the Banach fixed-point theorem then gives us a unique solution to equation \eqref{5-Zeroes-Equ-LEOperator}.\\
	Now let $0\leq n<1$.
	In this situation the preceding method does not work due to $x^n$ not being differentiable or Lipschitz at $x=0$. 
	Since by definition of $C^*$, we had chosen \eqref{5-Zeroes-Equ-LEOperator-Restrict-Slope} as an additional constraint, we can now choose $\epsilon$ small enough to have $\theta\geq q>0$ on $[0,\epsilon]$.
	Then the same procedure as before can be carried through but with the interval $I=[q,\theta]$.
\end{proof}\noindent
Equation \eqref{5-Zeroes-Equ-LEOperator-Derivative} shows that non-negative solutions must have non-positive slope.
The following theorem makes use of that fact and of a different representation of the \ac{LE} equation to show that solutions can be extended until they reach zero.
\begin{lemma}
	Let $\theta$ be a \ac{LE} solution and $[0,\xi_\mathrm{max}]$ be the maximum interval of existence.
	Then $\theta(\xi_\mathrm{max})=0$ or $\xi_\mathrm{max}=\infty$.
\end{lemma}
\begin{proof}
	It is clear that $\theta<0$ is a case that can never appear as long as we are interested in real solutions.
	Suppose $\theta(\xi_\mathrm{max})>0$ and $\xi_\mathrm{max}<\infty$.
	Then there exists a interval $(\xi_\mathrm{max}-\epsilon,\xi_\mathrm{max}+\epsilon)$ and $(\theta(\xi_\mathrm{max})-\delta,\theta(\xi_\mathrm{max})+\delta)$ where the \ac{LE} differential equation has a solution.
	With equations \ref{4-NumSol-Equ-LE-Substitution} we can directly see that the Picard-Lindelöf theorem is applicable in this situation.
\end{proof}
%
%
%
\subsection{TOV Equation}
\begin{hypothesis}[TOV Zero Value Hypothesis]
	Given the TOV differential equation with $\rho=Ap^{\frac{n}{n+1}}$ and $p_0,A>0$
	\begin{alignat}{3}
		\frac{\partial m}{\partial r} &= &&4\pi\rho r^2\\
		\frac{\partial p}{\partial r} &= -&&\frac{m\rho}{r^2}\left(1+\frac{p}{\rho}\right)\left(\frac{4\pi r^3p}{m}+1\right)\left(1-\frac{2m}{r}\right)^{-1}
		\label{5-Zeroes-Equ-TOV-Equ}
	\end{alignat}
	and initial values $m_0=0,p_0\geq0$ there exists a $n_0\geq0$ such that all solutions with same parameters $A,p_0$ and smaller exponent $n<n_0$ have a $p(r_0)$ for some $r_0>0$.
\end{hypothesis}
\begin{proof}
	...
	% TODO write something about this proof.
\end{proof}
\begin{lemma}
	\label{5-Zeroes-Lem-TOV-Conditions}
	Let $(m,p):\R\rightarrow\R^2$ be a \ac{TOV} solution with initial values $m_0=0$ and $p_0>0$.
	Also let $\rho:\R_{\geq0}\rightarrow\R_{\geq0}$ be the monotonously increasing \ac{eos}.
	Then for a small enough $\epsilon$ Interval $[0,\epsilon]$ the solution $f$ satisfies the following conditions.
	\begin{align}
		m(r) &\leq \frac{4\pi\rho_0r^3}{3}\label{5-Zeroes-Equ-TOV-Conditions-1}\\
		\lim_{r\rightarrow0}\frac{m}{r^3}&=\frac{4\pi\rho_0}{3}\label{5-Zeroes-Equ-TOV-Conditions-2}\\
		p(r)&\geq\frac{p_0}{2}\label{5-Zeroes-Equ-TOV-Conditions-3}
	\end{align}
\end{lemma}
\begin{proof}
	To prove the condition \eqref{5-Zeroes-Equ-TOV-Conditions-1} we inspect that with the initial value $m_0=0$ we can write
	\begin{equation}
		m = \int\limits_0^r 4\pi\rho r'^2dr'\leq4\pi\rho_0\int\limits r'^2dr'=\frac{4\pi\rho_0r^3}{3}
	\end{equation}
	where in the second step we used that $\rho$ is monotonously increasing and positive solutions of the \ac{TOV} equation have non-positive slope in the pressure $p$ component.\\
	The second condition is obtained when applying L'Hosptials Rule as was done in the beginning of section \ref{4-NumSol-Sec-Comp-TOV-LE}.
	For the third condition we first inspect that condition \ref{5-Zeroes-Equ-TOV-Conditions-1} lets us choose $(1-2m/r)^{-1}\leq 2$.
	\begin{align}
		p_0 - p &=\int\limits_0^r\frac{1}{r'^2}\left(p+\rho\right)\left(4\pi pr'^3+m\right)\left(1-\frac{2m}{r'}\right)^{-1}dr'\\
				&\leq2\left(p_0+\rho_0\right)\int\limits_0^r\frac{4\pi pr'^3+m}{r'^2}dr'\\
				&\leq2(p_0+\rho_0)\int\limits_0^r\left(4\pi p_0r' + \frac{4\pi\rho_0r'}{3}\right)dr'\\
				&=4\pi(p_0+\rho_0)\left(p_0+\frac{\rho_0}{3}\right)r^2
	\end{align}
	In the first step we used that $p$ is a \ac{TOV} solution while in the second step the aforementioned inequality and just as in the last step the non-positive slope of $p$ with the monotonicity of $\rho$ came into play.
	Afterwards we used the first condition \ref{5-Zeroes-Equ-TOV-Conditions-1} and integrated the result.
	It is now clear that a small enough $\epsilon>0$ can be chosen such that all stated conditions are satisfied.
\end{proof}
\begin{lemma}[TOV Local existence]
	The \ac{TOV} equation \ref{5-Zeroes-Equ-TOV-Equ} with initial values $m_0=0$ and $p_0\geq0$ and monotonously increasing \ac{eos} $\rho:\R_{\geq0}\rightarrow\R_{\geq0}$ has a local solution for $r\in[0,\epsilon]$ with some $\epsilon>0$.
\end{lemma}
\begin{proof}
	The proof follows similar principles as previously done in lemma \ref{5-Zeroes-Lem-Lane-EmdenFiniteBoundary1}.
	In order to solve the problem at hand we need to define a space $K$ consisting of all functions $f:[0,\epsilon]\rightarrow\R_{\geq0}^2$ that all satisfy the same boundary conditions in form of initial values $m_0=0$ and $p_0>0$.
	Additionally we have shown in lemma \ref{5-Zeroes-Lem-TOV-Conditions} that we can apply the listed conditions for a small enough interval $[0,\epsilon]$ without reducing the number of possible solutions.
	Since solutions of the \ac{TOV} equation have monotonously decreasing pressure $p$ and monotonously increasing mass $m$ those are taken as an additional requirement.
	We now call $K$ the space of all functions $f:[0,\epsilon]\rightarrow\R_{\geq0}^2$ that satisfy these conditions.
	This space is complete with the norm $||.||_\infty$.
	We will now again apply the Banach fixed-point theorem in combination with lemma \ref{5-Zeroes-Lem-Integral-Contraction} to obtain a unique solution.
	Let $T:K\rightarrow K$ be defined by the \ac{TOV} equation
	\begin{equation}
		T((m,p))(r)=\left(\int\limits_0^r 4\pi\rho r'^2dr',-\int\limits_0^r\frac{1}{r'^2}(4\pi\rho r'^3 + m)\left(1-\frac{2m}{r'}\right)^{-1}dr'\right).
	\end{equation}
	Now clearly the inner arguments of the integral are differentiable in $r,m,p$ and well defined if $r>0$.
	There are some terms which cause problems when $r=0$.
	Looking at $m/r$ we see that condition \eqref{5-Zeroes-Equ-TOV-Conditions-2} we issued now tells us that $m/r\rightarrow0$ as $r\rightarrow0$.
	Similarly condition \eqref{5-Zeroes-Equ-TOV-Conditions-2} implies that $m/r^2\rightarrow0$ in the same limit.
	This now means that on this space $K$ the integrand has a well defined differential for every $r\in[0,\epsilon]$ which in turn shows with lemma \ref{5-Zeroes-Lem-Lipschitz-Continuity-Differentiable} that it is Lipschitz.
	Using lemma \ref{5-Zeroes-Lem-Integral-Contraction} we can apply the Banach fixed-point theorem and obtain a unique solution.
\end{proof}
%
%
%
