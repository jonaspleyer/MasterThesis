\section{To a General Zero Value Theorem}
\label{sec:5-zeroes}
\subsection{\texorpdfstring{\acrNoHyperlink{\acs}{LE}}{LE} Equation}
\label{subsec:5-zeroes-le-equation}
Motivated by the preceding sections, the aim now is to prove the following theorem.
\begin{theorem}[\acrNoHyperlink{\acl}{LE} Finite Boundary]
	\label{5-Zeroes-Theo-Lane-EmdenFiniteBoundary}
	For every $0\leq n<5$, the generalised Lane-Emden equation
	\begin{equation}
		\frac{1}{\xi^{l-1}}\frac{d}{d\xi}\left(\xi^{l-1}\frac{d\theta}{d\xi}\right)+\theta^n=0
		\label{eq:5-Zeroes-Equ-LEeq}
	\end{equation}
	with $\theta_0=1$ and $l\geq2$ has a zero value for a finite $\xi_0$.
	To prove this theorem, we show the statements following in this section.
\end{theorem}
\begin{lemma}[\acrNoHyperlink{\acs}{LE} Local Existence]
	\label{5-Zeroes-Lem-Lane-Emden-Local-Existence}
	For every $n\geq0$, the Lane-Emden equation~\ref{eq:5-Zeroes-Equ-LEeq} with initial values defined as above has a unique solution in an $\epsilon$ Ball around $\xi=0$.
\end{lemma}\noindent
The procedure for proving existence is similar for both equations.
In a similar approach given by the author in~\cite[p.~50]{quittnerSuperlinearParabolicProblems2007}, we want to use the Schauder fixed-point theorem~\cite{schauderFixpunktsatzFunktionalraumen1930}, which is a variation the Banach fixed-point theorem to show that over a suitably chosen space of functions we obtain a unique solution to the differential equation.
The statement as formulated in~\cite{minazzoTheoremesPointFixe2007} reads
\begin{theorem}
	Every continuous function from a convex compact subset K of a Banach space to K itself has a fixed point.
\end{theorem}\noindent
In order to follow this approach, we examine some properties that solutions of the \ac{LE} differential equation will have.
\begin{lemma}
	\label{5-Zeroes-Lem-LE-Conditions}
	A positive solution $\theta$ of the \ac{LE} equation with $n\geq0$ on a suitably chosen interval $[0,\lambda]$ obeys the conditions
	\begin{align}
		\theta(0) 					&= \theta_0>0\label{5-Zeroes-Equ-LE-Conditions-Initial-1}\\
		\left.\frac{d\theta}{d\xi}\right|_{\xi=0} &= 0 \label{5-Zeroes-Equ-LE-Conditions-Initial-2}\\
		x\leq y 					&\Rightarrow \theta(y)\leq\theta(x)\label{5-Zeroes-Equ-LE-Conditions-3}\\
		\theta_0/2					&\leq \theta \label{5-Zeroes-Equ-LE-Conditions-4}\\
		|\theta(x)-\theta(y)|		&\leq L|x-y|\label{5-Zeroes-Equ-LE-Conditions-5}
	\end{align}
	where $L$ is independent of $\theta$.
	We will call $C^*\subseteq C^0$ the space of all continuous functions $f:[0,\lambda]\rightarrow\R$ that satisfy those conditions.
\end{lemma}
\begin{proof}
	The first and second initial conditions of equations~\eqref{5-Zeroes-Equ-LE-Conditions-Initial-1} and~\eqref{5-Zeroes-Equ-LE-Conditions-Initial-2} are true by assumption.
	The third condition~\eqref{5-Zeroes-Equ-LE-Conditions-3} can be obtained by integrating the \ac{LE} equation once
	\begin{equation}
		\frac{d\theta}{d\xi} = -\int\limits_0^\xi \frac{t^{l-1}}{\xi^{l-1}}\theta^n \diff t
		\label{eq:5-Zeroes-LE-Equation-Rewritten-Integration-1}
	\end{equation}
	where we not yet needed any of the above criteria except positivity of $\theta$.
	% From this expression, condition~\eqref{5-Zeroes-Equ-LE-Conditions-5} is derived immediately.
	When integrating equation~\eqref{eq:5-Zeroes-LE-Equation-Rewritten-Integration-1} once more and using the initial value of condition~\eqref{5-Zeroes-Equ-LE-Conditions-Initial-1}, we obtain
	\begin{equation}
		\theta(\xi) = \theta_0 -\int\limits_0^\xi\int\limits_0^s \left(\frac{t}{s}\right)^{l-1}\theta^n(t) \diff t \diff s.
		\label{eq:5-Zeroes-LE-Equation-Rewritten-Integration-2}
	\end{equation}
	We combine this statement with the previously shown condition of equation~\eqref{5-Zeroes-Equ-LE-Conditions-3}.
	Then it is clear that
	\begin{equation}
		\int\limits_0^\xi\int\limits_0^s \left(\frac{t}{s}\right)^{l-1}\theta^n(t) \diff t \diff s \leq \theta_0^n\int\limits_0^\xi\int\limits_0^s \left(\frac{t}{s}\right)^{l-1} \diff t \diff s \leq \theta_0^n\frac{\xi^2}{2l}
		\label{eq:5-Zeroes-LE-Equation-Rewritten-Inequality}
	\end{equation}
	and thus $\xi\leq\lambda$ can be chosen suitably small to satisfy condition~\eqref{5-Zeroes-Equ-LE-Conditions-4}.
	For the last condition we derive from equation~\eqref{eq:5-Zeroes-LE-Equation-Rewritten-Integration-2} that
	\begin{equation}
		|\theta(x)-\theta(y)| = \left|\int\limits_x^y\int\limits_0^s \left(\frac{t}{s}\right)^{l-1}\theta^n(t) \diff t \diff s\right| \leq \theta^n_0\frac{|x^2-y^2|}{2l} \leq \theta^n_0\frac{\lambda}{l}|x-y|
		\label{eq:5-Zeroes-LE-Equation-Rewritten-Inequality-2}
	\end{equation}
\end{proof}\noindent
The now introduced function space $C^*$ contains all functions that are possible solutions for the \ac{LE} equation.
In order to use the Schauder fixed-point theorem, we have to show that $C^*$ is compact and convex which we will show in the next two lemmata.
\begin{lemma}
	The space $C^*$ defined in lemma~\ref{5-Zeroes-Lem-LE-Conditions} with the norm $||\cdot||_\infty$ is complete and convex.
	\label{5-Zeroes-Lem-LE-Convex-Complete}
\end{lemma}
\begin{proof}
	We begin by showing completeness.
	It is clear that conditions~\eqref{5-Zeroes-Equ-LE-Conditions-Initial-1} to~\eqref{5-Zeroes-Equ-LE-Conditions-4} are conserved.
	Generally speaking Lipschitz continuous functions need not be complete under $||\cdot||_\infty$.
	However, since we had chosen $L$ to be independent of $\theta$ for $||f_m-f||_\infty\rightarrow0$ we can write
	\begin{equation}
		|f(x)-f(y)|=|f(x)-f_m(x)+f_m(x)-f_m(y)+f_m(y)-f(y)|\leq 2\epsilon+L|x-y|
		\label{eq:5-Zeroes-LE-Equicontinuity-Proof}
	\end{equation}
	which proves the statement in the limit $\epsilon\rightarrow0$.
	Convexity for conditions~\eqref{5-Zeroes-Equ-LE-Conditions-Initial-1} to~\eqref{5-Zeroes-Equ-LE-Conditions-4} is also clear.
	Let $f,g\in C^*$ and $h\in[0,1]$.
	Then
	\begin{alignat}{3}
		&&&|hf(x)+(1-h)g(x)-hf(y)-(1-h)g(y)|\\
		&\leq h&&|f(x)-f(y)|+ (1-h)|g(x)-g(y)|\\
		&\leq L&&|x-y|
	\end{alignat}
	which shows convexity.
\end{proof}\noindent
\begin{lemma}
	The space $C^*$ with $||\cdot||_\infty$ is compact.
	\label{5-Zeroes-Lem-LE-Compact}
\end{lemma}
\begin{proof}
	To prove part of this statement we use the Arzela-Ascoli theorem.
	Since $C^*$ is a function space over a compact set $[0,\lambda]$ we only need to show that it is bounded and equicontinuous.
	For every function $f\in C^*$ we know by condition~\eqref{5-Zeroes-Equ-LE-Conditions-3} and~\eqref{5-Zeroes-Equ-LE-Conditions-4} that $\theta/2\leq f\leq\theta_0$ which shows boundedness.
	Since with condition~\eqref{5-Zeroes-Equ-LE-Conditions-5} every function $f\in C^*$ is especially Lipschitz, we can clearly see that equicontinuity hols.
\end{proof}
\begin{proof}[Proof \acrNoHyperlink{\acs}{LE} Local Existence]
	With these results it is now possible to prove our initial statement of theorem~\ref{5-Zeroes-Lem-Lane-Emden-Local-Existence}.
	It is left to prove that the operator $T:C^*\rightarrow C^*$ defined by
	\begin{equation}
		T(f)(x) = \theta_0 - \int\limits_0^x\int\limits_0^s \left(\frac{t}{s}\right)^{l-1}f^n(t)\diff t\diff s
		\label{eq:5-Zeroes-LE-Definition-Operator-On-C-Star}
	\end{equation}
	is continuous on $C^*$.
	With a quick integral approximation we obtain
	\begin{align}
		||T(f)-T(g)||_{\infty} &= ||\int\limits_0^x\int\limits_0^s \left(\frac{t}{s}\right)^{l-1}\left[f^n(t)-g^n(t)\right]\diff t\diff s\\
		&\leq \frac{x^2}{2l}||f^n-g^n||_{\infty}.
	\end{align}
	Since $z\mapsto z^n$ is continuous for $z\geq0$ and $n\geq0$ with $\mathrm{img}(f)\subseteq[\theta_0/2,\theta_0]$ for each $f\in C^*$ by condition~\eqref{5-Zeroes-Equ-LE-Conditions-4} the statement follows directly.
\end{proof}\noindent
Equation~\eqref{eq:5-Zeroes-LE-Equation-Rewritten-Integration-1} shows that non-negative solutions must have non-positive slope.
The following theorem makes use of that fact and of a different representation of the \ac{LE} equation to show that solutions can be extended until they reach zero.
\begin{lemma}
	Let $\theta$ be a \ac{LE} solution and $[0,\xi_\textrm{m}]$ be the maximum interval of existence.
	Then $\theta(\xi_\textrm{m})=0$ or $\xi_\textrm{m}=\infty$.
\end{lemma}
\begin{proof}
	It is clear that $\theta<0$ is a case that can never appear as long as we are interested in positive solutions.
	Suppose $\theta(\xi_\textrm{m})>0$ and $\xi_\textrm{m}<\infty$.
	Then there exists an interval $(\xi_\textrm{m}-\epsilon,\xi_\textrm{m}+\epsilon)$ and $(\theta(\xi_\textrm{m})-\delta,\theta(\xi_\textrm{m})+\delta)$ where the \ac{LE} differential equation has a solution.
	With equations~\ref{eq:4-NumSol-Equ-LE-Substitution} we can directly see that the Picard-Lindelöf theorem~\cite{lindelofApplicationMethodeApproximations1894} is applicable in this situation.
\end{proof}
\begin{proof}[Proof \acrNoHyperlink{\acl}{LE} Finite Boundary]
	To prove Theorem~\ref{5-Zeroes-Theo-Lane-EmdenFiniteBoundary} it is left to show that the \ac{LE} equation has no global solutions for $n<5$.
	In~\cite[p.~36]{quittnerSuperlinearParabolicProblems2007}, the author shows that for $n\leq5$, \ac{LE} solutions can not be global solutions.
	Furthermore it is explicitly shown that for $n\geq5$ global solutions and thus no zero-value exist.
	A variation of the proof for this particular usecase can be found in appendix~\ref{subsec:99-App-NoGlobalLE}.
\end{proof}
%
%
\subsection{\texorpdfstring{\acrNoHyperlink{\acs}{TOV}}{TOV} Equation}
\label{subsec:5-Zeroes-TOV-Equ}
Motivated by the previous section about numerical solutions, we state the following hypothesis.
\begin{hypothesis}[\acrNoHyperlink{\acs}{TOV} Zero Value Hypothesis]
	\label{theo:5-Zeroes-TOV-Zero-Value-Hypothesis}
	Given the \ac{TOV} differential equation with polytropic \ac{eos} $\rho=Ap^{\frac{n}{n+1}}$ and $p_0,A>0$
	\begin{alignat}{3}
		\frac{\partial m}{\partial r} &= &&4\pi\rho r^2\\
		\frac{\partial p}{\partial r} &= -&&\frac{m\rho}{r^2}\left(1+\frac{p}{\rho}\right)\left(\frac{4\pi r^3 p}{m}+1\right)\left(1-\frac{2m}{r}\right)^{-1}
		\label{5-Zeroes-Equ-TOV-Equ}
	\end{alignat}
	and initial values $m_0=0,p_0\geq0$ there exists a $n_0\geq0$ such that all solutions with same parameters $A,p_0$ and smaller exponent $0\leq n<n_0$ have $p(r_0)$ for some $r_0>0$.
\end{hypothesis}
% TODO write something about this proof. First look at the proof given in quittner
\begin{lemma}[\acrNoHyperlink{\acs}{TOV} Local existence]
	The \ac{TOV} equation~\ref{5-Zeroes-Equ-TOV-Equ} with initial values $m_0=0$ and $p_0\geq0$ and monotonously increasing continuous \ac{eos} $\rho:\R_{\geq0}\rightarrow\R_{\geq0}$ with $\rho(0)=0$ has a local solution for $r\in[0,\epsilon]$ with some $\epsilon>0$.
\end{lemma}
\begin{lemma}
	\label{5-Zeroes-Lem-TOV-Conditions}
	Let $(m,p):\R\rightarrow\R^2$ be a \ac{TOV} solution with initial values $m_0=0$ and $p_0>0$.
	Also let $\rho:\R_{\geq0}\rightarrow\R_{\geq0}$ be the polytropic \ac{eos}.
	Then for a small enough $\epsilon$ Interval $[0,\epsilon]$ the solution $(m,p)$ satisfies the following conditions.
	\begin{align}
		p(0)=p_0>0 &\mathrm{\ and\ } m(0)=m_0=0\label{5-Zeroes-Equ-TOV-Conditions-0}\\
		m(r) &\leq \frac{4\pi\rho_0 r^3}{3}\label{5-Zeroes-Equ-TOV-Conditions-1}\\
		\lim_{r\rightarrow0}\frac{m}{r^3}&=\frac{4\pi\rho_0}{3}\label{5-Zeroes-Equ-TOV-Conditions-2}\\
		p(r)&\geq\frac{p_0}{2}\label{5-Zeroes-Equ-TOV-Conditions-3}\\
		x\leq y&\Rightarrow m(x)\leq m(y)\label{5-Zeroes-Equ-TOV-Conditions-4}\\
		x\leq y&\Rightarrow p(x)\geq p(y)\label{5-Zeroes-Equ-TOV-Conditions-5}\\
		|m(x)-m(y)|+|p(x)-p(y)|&\leq L|x-y|\forall x,y\in[0,\epsilon]\forall(m,p)\in K\label{5-Zeroes-Equ-TOV-Conditions-6}
	\end{align}
\end{lemma}
\begin{proof}
	Conditions~\eqref{5-Zeroes-Equ-TOV-Conditions-0},\eqref{5-Zeroes-Equ-TOV-Conditions-4} and~\eqref{5-Zeroes-Equ-TOV-Conditions-5} are clear by looking at equation~\eqref{5-Zeroes-Equ-TOV-Equ}.
	To prove condition~\eqref{5-Zeroes-Equ-TOV-Conditions-1}, we inspect that with the initial value $m_0=0$ we can write
	\begin{equation}
		m = \int\limits_0^r 4\pi\rho r'^2 \diff r'\leq4\pi\rho_0\int\limits_0^r r'^2 \diff r'=\frac{4\pi\rho_0 r^3}{3}
		\label{eq:5-Zeroes-TOv-Conditions-Proof-Intermed-1}
	\end{equation}
	where in the second step we used that $\rho$ is monotonously increasing and positive solutions of the \ac{TOV} equation have non-positive slope in the pressure $p$ component.\\
	The third condition is obtained when applying L'Hosptial's Rule as was done in the beginning of section~\ref{subsec:4-NumSol-Sec-Comp-TOV-LE}.
	For the third condition we first inspect that condition~\ref{5-Zeroes-Equ-TOV-Conditions-1} lets us choose $(1-2m/r)^{-1}\leq 2$.
	\begin{align}
		p_0 - p &=\int\limits_0^r\frac{1}{r'^2}\left(p+\rho\right)\left(4\pi pr'^3+m\right)\left(1-\frac{2m}{r'}\right)^{-1}\diff r'\label{5-Zeroes-Equ-TOV-Conditions-Proof-1}\\
				&\leq2\left(p_0+\rho_0\right)\int\limits_0^r\frac{4\pi pr'^3+m}{r'^2}\diff r'\label{5-Zeroes-Equ-TOV-Conditions-Proof-2}\\
				&\leq2(p_0+\rho_0)\int\limits_0^r\left(4\pi p_0 r' + \frac{4\pi\rho_0 r'}{3}\right)\diff r'\label{5-Zeroes-Equ-TOV-Conditions-Proof-3}\\
				&=4\pi(p_0+\rho_0)\left(p_0+\frac{\rho_0}{3}\right)r^2\label{5-Zeroes-Equ-TOV-Conditions-Proof-4}
	\end{align}
	In the first step we used that $p$ is a \ac{TOV} solution while in the second step the aforementioned inequality and just as in the last step the non-positive slope of $p$ with the monotonicity of $\rho$ came into play.
	Next we used the condition~\ref{5-Zeroes-Equ-TOV-Conditions-1} to restrict $r\in[0,\epsilon]$ such that $(1-2m/r)^{-1}\leq2$ and integrated the result.
	Condition~\eqref{5-Zeroes-Equ-TOV-Conditions-6} can be shown when inspecting $m$ and $p$ individually.
	\begin{align}
		|m(x)-m(y)|&=\left|\int\limits_x^y 4\pi\rho r'^2 \diff r'\right|\leq4\pi\rho_0\epsilon^2|x-y|\\
		|p(x)-p(y)|&\leq8\epsilon\pi(p_0+\rho_0)\left(p_0+\frac{\rho_0}{3}\right)|x-y|
	\end{align}
	For the first equation we used the decreasing monotonous behaviour of $p$ and increasing behaviour of $\rho$.
	Afterwards, the just proven lines~\eqref{5-Zeroes-Equ-TOV-Conditions-Proof-1} to~\eqref{5-Zeroes-Equ-TOV-Conditions-Proof-4} and $x,y\leq\epsilon$ show the next statement.
	It is now clear that a small enough $\epsilon>0$ can be chosen such that all stated conditions are satisfied.
\end{proof}\noindent
We will now call $K$ the space of functions $f:[0,\epsilon]\rightarrow\R_{\geq0}^2$ that obey conditions~\eqref{5-Zeroes-Equ-TOV-Conditions-0} to~\eqref{5-Zeroes-Equ-TOV-Conditions-6}.
\begin{lemma}
	\label{5-Zeroes-Lem-K-Complete-Convex}
	The space $K$ with $||\cdot||_\infty$ is complete and convex.
\end{lemma}
\begin{proof}
	We need to show that conditions~\eqref{5-Zeroes-Equ-TOV-Conditions-0} to~\eqref{5-Zeroes-Equ-TOV-Conditions-6} are preserved under convexity and convergence in the $||\cdot||_\infty$ norm.
	We only need to show this for equation~\eqref{5-Zeroes-Equ-TOV-Conditions-6} which is true by the same argument as in the \ac{LE} case in lemma~\ref{5-Zeroes-Lem-LE-Convex-Complete}.
\end{proof}
\begin{lemma}
	The space $K$ is compact.
\end{lemma}
\begin{proof}
	We again use the Arzela-Ascoli Theorem.
	By condition~\eqref{5-Zeroes-Equ-TOV-Conditions-1} and $r\in[0,\epsilon]$ it is clear that all functions $f\in K$ are bounded.
	To show equicontinuity, we can apply the exact same argument as in lemma~\ref{5-Zeroes-Lem-LE-Compact} with condition~\eqref{5-Zeroes-Equ-TOV-Conditions-6} to obtain the desired result.
\end{proof}
\begin{proof}[Proof \acrNoHyperlink{\ac}{TOV} Existence]
	To show that local solutions of the \ac{TOV} equation exist, it remains to be proven that $U:K\rightarrow K$ defined by
	\begin{align}
		U_1((m,p))(r) &= \int\limits_0^r 4\pi\rho(p) r'^2 \diff r'
		\label{eq:5-Zeroes-TOV-Operator-Def-1}\\
		U_2((m,p))(r) &= -\int\limits_0^r\frac{p+\rho(p)}{r'^2}(4\pi\rho r'^3 + m)\left(1-\frac{2m}{r'}\right)^{-1}\diff r'
		% TODO This equation has overfull hbox. Fix it!
		\label{eq:5-Zeroes-TOV-Operator-Def-2}.
	\end{align}
	is continuous.
	For $(m,p),(u,q)\in K$ we respectively inspect the first and second component.
	By continuity of $\rho$, we have $|\rho(p)-\rho(q)|<\varepsilon$.
	Since by condition~\eqref{5-Zeroes-Equ-TOV-Conditions-0} and~\eqref{5-Zeroes-Equ-TOV-Conditions-3}, we know that we only need to consider $\rho$ on a compact interval $[p_0,p_0/2]$, it is clear that by choosing $||p-q||<\delta$ small enough the term in equation~\eqref{eq:5-Zeroes-TOV-Ineq-1} shrinks arbitrarily.
	\begin{equation}
		\left|\left|\int\limits_0^r 4\pi(\rho(p)-\rho(q))r'^2\diff r\right|\right|_\infty\leq \frac{4\pi r^3}{3}||\rho(p)-\rho(q)||_\infty.
		%\left|\left|U((m,p))-U((u,q))\right|\right|_{\infty} \leq
		\label{eq:5-Zeroes-TOV-Ineq-1}
	\end{equation}
	For the second component, we make use of condition~\eqref{5-Zeroes-Equ-TOV-Conditions-1} to again restrict $r\in[0,\epsilon]$ such that $(1-2m/r)^{-1}\leq2$.
	\begin{align}
		\left|\left|U_2(m,p)-U_2(u,q)\right|\right|_\infty &\leq \int\limits_0^r(p_0+\rho_0)\left|\left|\left(4\pi r'(\rho(p))-\rho(q))+\frac{m-u}{r'^2}\right)\right|\right|_\infty\diff r'\\
		&\leq (p_0+\rho_0)M\left|\left|p-q\right|\right|_\infty + 8\pi\rho_0\epsilon(p_0+\rho_0)\left|\left|\frac{m-u}{r^2}\right|\right|_\infty
		\label{eq:5-Zeroes-TOV-Ineq-2}
	\end{align}
	By condition~\eqref{5-Zeroes-Equ-TOV-Conditions-2} we see that the term
	\begin{equation}
		\frac{m-u}{r^2}\rightarrow0
		\label{eq:5-Zeroes-TOV-Limit-UM-Div-R2}
	\end{equation}
	in the limit $r\rightarrow0$ and in general is bound by $8\pi\epsilon\rho_0/3$.
	This shows that for every $m,u$ there must exist a $\tau_0>0$ such that~\eqref{eq:5-Zeroes-TOV-Limit-UM-Div-R2} has its maximum value, or it is identically $0$.
	\begin{equation}
		\left|\left|\frac{m-u}{r^2}\right|\right|_{\infty} = \frac{1}{\tau_0^2}|m(\tau_0)-u(\tau_0)|\leq\frac{1}{\tau_0^2}||m-u||_{\infty}
		\label{eq:5-Zeroes-TOV-Last-Term}
	\end{equation}
	Let $m_k\rightarrow u$ be a series in $(K.||\cdot||_\infty)$.
	% Suppose that for only finite elements of this series the maximum is $0$ and remove them from this series.
	Without loss of generality, we can remove all elements of the series for which the maximum is identically $0$ and still have more than finite elements left.
	% Otherwise, there would exist a subsequence with
	% \begin{equation}
	% 	\left|\left|\frac{m_{k_i}-u}{r^2}\right|\right|_{\infty}=0
	% 	\label{eq:5-Zeroes-TOV-Last-Term-Series}
	% \end{equation}
	% for more than finite $k_i\in\mathbb{N}$ and the statement would be proven.
	One can obtain another series $\tau_k$ by selecting the maximum point as done in equation~\eqref{eq:5-Zeroes-TOV-Last-Term}.
	This series is bound by assumption in $[0,\epsilon]$ and thus has a convergent subsequence.
	Consider $\tau_k\rightarrow\tau$.
	Then by definition we have
	\begin{equation}
		\left|\left|\frac{m_k-u}{r^2}\right|\right|_{\infty} = \frac{1}{\tau_k^2}\left|m_k(\tau_k)-u(\tau_k)\right| \leq \frac{1}{\tau_k^2} \left|\left|m_k-u\right|\right|_{\infty}.
		\label{eq:5-Zeroes-TOV-Last-Term-Series-2}
	\end{equation}
	If $\tau=0$, condition~\eqref{5-Zeroes-Equ-TOV-Conditions-2} implies that equation~\eqref{eq:5-Zeroes-TOV-Last-Term-Series-2} vanishes in the limit.
	In the case $\tau>0$, we can approximate equation~\eqref{eq:5-Zeroes-TOV-Limit-UM-Div-R2} by taking the smallest $\tau_{j}>0$ of the series and the result is proven again.
	This proves continuity of $T$ which in turn shows local existence of \ac{TOV} solutions.
\end{proof}%
% Constructing a proof of~\ref{theo:5-Zeroes-TOV-Zero-Value-Hypothesis}.