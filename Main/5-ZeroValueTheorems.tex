\section{To a General Zero Value Theorem}
\subsection{LE Equation}
Motivated by the preceding sections, the aim now is to prove the following theorem.
\begin{theorem}[Lane-Emden Finite Boundary]
	\label{Theo-Lane-EmdenFiniteBoundary}
	For every $0\leq n\leq5$, the Lane-Emden equation
	\begin{equation}
		\frac{1}{\xi^2}\frac{d}{d\xi}\left(\xi^2\frac{d\theta}{d\xi}\right)+\theta^n=0
		\label{5-Ideas-LEeq}
	\end{equation}
	with $\theta_0=1$ has a zero value for a finite $\xi_0$. To prove this theorem, show the statements \ref{Theo-Lane-EmdenFiniteBoundary1}, 
	\ref{Theo-Lane-EmdenFiniteBoundary2} and \ref{Theo-Lane-EmdenFiniteBoundary3}.
\end{theorem}
We explain what the idea behind the proof is.
...
\begin{lemma}[LE Local Existence]
	\label{Theo-Lane-EmdenFiniteBoundary1}
	For every $n\geq0$, the Lane-Emden equation \ref{5-Ideas-LEeq} has a solution in an $\epsilon$ Ball around $\xi=0$.
\end{lemma}
\begin{proof}
	This proof can be generalised to $m\geq2$ dimensions. 
	Following \cite{quittnerSuperlinearParabolicProblems2007a} we see that the LE differential equation can be written as
	\begin{equation}
		\frac{d}{d\xi}\left(\xi^{m-1}\frac{d\theta}{d\xi}\right) + \xi^{m-1}\theta^n = 0.
	\end{equation}
	Integrating once and rearranging leads us to
	\begin{equation}
		\frac{d\theta}{d\xi} = \left.\frac{d\theta}{d\xi}\right|_0 - \int\limits_0^\xi\left(\frac{t}{\xi}\right)^{m-1}\theta^ndt.
	\end{equation}
	By choosing the initial value of $d\theta/d\xi=0$ we see that positive solutions to the LE equation need to have negative slope.
	This equation also shows that for $\theta_0>0$, the solution $\theta$ will be positive in a $\epsilon$ region around $\xi=0$.
	We integrate both sides again and end up with
	\begin{equation}
		\theta = \theta_0 - \int\limits_0^\xi\int\limits_0^s\left(\frac{t}{s}\right)^{m-1}\theta^ndtds.
		\label{5-Ideas-LEOperator}
	\end{equation}
	We define $C^*$ as the space of all real continuous functions $f\geq0$ on $[0,\epsilon]$ that share the same initial value $\theta_0$ and with negative slope. %where $n\in\R$ refers to the exponent given before.
	With the norm $||.||_\infty$ this space is complete.
	Inspecting the operator $T:C^*\rightarrow C^*$ defined by
% 	\footnote{The  index $1$ refers to the $C^1$ norm given by $||f||_2=||f||_\infty+||f'||_\infty+||f''||_\infty$ on the yet to be defined space.}
	equation \ref{5-Ideas-LEOperator} we can apply the Banach fixed-point theorem.
% 	Without loss of generality, we can assume $U\neq V$. Then
	\begin{align}
		||T(U)-T(V)||_\infty	&\leq	\left|\left|U^n-V^n\right|\right|_\infty\int\limits_0^\xi\int\limits_0^s \left(\frac{t}{s}\right)^{m-1}dtds\\
								&=		\left|\left|U^n-V^n\right|\right|_\infty\frac{\xi^2}{2m}
	\end{align}
	Let $n\geq1$. 
	Since the function $x\mapsto x^n$ is differentiable for $n\geq1$ it is also Lipschitz with constant $L$ on the compact interval $I=[0,\theta_0]$.
	This means in particular that we have
	\begin{equation}
		||T(U)-T(V)||_\infty	\leq	\left|\left|U-V\right|\right|_\infty\frac{L\xi^2}{2m}
	\end{equation}
	Thus when choosing $\epsilon<\sqrt{2m/L}$ we have a contraction and the Banach fixed-point theorem then gives us a unique solution to equation \ref{5-Ideas-LEOperator}.\\
	Now let $0\leq n<1$.
	In this situation the preceding method does not work due to $x^n$ not being differentiable at $x=0$. 
	For $\theta_0\leq1$, we know that $\theta^n\leq\theta_0^n$. We can now choose $\epsilon$ such that with
	\begin{equation}
		\int\limits_0^\xi\int\limits_0^s\left(\frac{t}{s}\right)^{m-1}\theta^n(t)dtds \leq \frac{\xi^2}{2m}\theta_0^n
	\end{equation}
	we have $\theta\geq\theta_0/2$ on $[0,\epsilon]$. If $\theta_0>1$, we have $\theta^n\leq\theta_0$ and obtain the same result. 
	Then the same procedure as before can be carried through but with the interval $I=[\theta_0/2,\theta]$.
\end{proof}
Maybe talk about more details or how we are proceding.

\begin{lemma}
	
\end{lemma}

%
%
%
\subsection{TOV Equation}

%
%
%
