\section{To a General Zero Value Theorem}
\subsection{LE Equation}
Motivated by the preceding sections, the aim now is to prove the following theorem.
\begin{theorem}[Lane-Emden Finite Boundary]
	\label{5-Zeroes-Lane-EmdenFiniteBoundary}
	For every $0\leq n<5$, the Lane-Emden equation
	\begin{equation}
		\frac{1}{\xi^2}\frac{d}{d\xi}\left(\xi^2\frac{d\theta}{d\xi}\right)+\theta^n=0
		\label{5-Ideas-LEeq}
	\end{equation}
	with $\theta_0=1$ has a zero value for a finite $\xi_0$. To prove this theorem, show the statements \ref{5-Zeroes-Lane-EmdenFiniteBoundary1}, 
	\ref{5-Zeroes-Lane-EmdenFiniteBoundary2} and \ref{5-Zeroes-Lane-EmdenFiniteBoundary3}.
\end{theorem}
We explain what the idea behind the proof is.
...
\begin{lemma}[LE Local Existence]
	\label{5-Zeroes-Lane-EmdenFiniteBoundary1}
	For every $n\geq0$, the Lane-Emden equation \ref{5-Ideas-LEeq} with initial values defined as above has a unique solution in an $\epsilon$ Ball around $\xi=0$.
\end{lemma}
With the aid of the Picard-Lindelöf theorem or the 

\begin{proof}
	This proof can be generalised to $l\geq2$ dimensions. 
	Following \cite{quittnerSuperlinearParabolicProblems2007a} we see that the LE differential equation can be written as
	\begin{equation}
		\frac{d}{d\xi}\left(\xi^{l-1}\frac{d\theta}{d\xi}\right) + \xi^{l-1}\theta^n = 0.
	\end{equation}
	Integrating once and rearranging leads us to
	\begin{equation}
		\frac{d\theta}{d\xi} = \left.\frac{d\theta}{d\xi}\right|_0 - \int\limits_0^\xi\left(\frac{t}{\xi}\right)^{l-1}\theta^ndt.
		\label{5-Zeroes-LEOperator-Derivative}
	\end{equation}
	By choosing the correct initial value\footnote{One has to choose this value in order to obtain the correct limiting case when using the non-relativistic limit of the \ac{TOV} equation. See also section \ref{3-Mass-LEDerivation}} of $d\theta/d\xi=0$ we see that positive solutions to the LE equation need to have negative slope.
	This equation also shows that for $\theta_0>0$, the solution $\theta$ will be positive in a $\epsilon$ region around $\xi=0$.
	We integrate both sides again and end up with
	\begin{equation}
		\theta = \theta_0 - \int\limits_0^\xi\int\limits_0^s\left(\frac{t}{s}\right)^{l-1}\theta^ndtds.
		\label{5-Zeroes-LEOperator}
	\end{equation}
	We define $C^*$ as the space of all real continuous functions $f\geq0$ on $[0,\epsilon]$ that share the same initial value $\theta_0$ and with non-positive slope. %where $n\in\R$ refers to the exponent given before.
	Additionally with equation \eqref{5-Zeroes-LEOperator-Derivative} we can restrict ourselves to functions with
	\begin{equation}
		\frac{d\theta}{d\xi} \geq -\theta_0^n\frac{\xi}{m}.
		\label{5-Zeroes-LEOperator-Restrict-Slope}
	\end{equation}
	With the norm $||.||_\infty$ this space is complete.
	Inspecting the operator $T:C^*\rightarrow C^*$ defined by
% 	\footnote{The  index $1$ refers to the $C^1$ norm given by $||f||_2=||f||_\infty+||f'||_\infty+||f''||_\infty$ on the yet to be defined space.}
	equation \ref{5-Zeroes-LEOperator} we can apply the Banach fixed-point theorem \cite{banachOperationsDansEnsembles1922}.
% 	Without loss of generality, we can assume $U\neq V$. Then
	\begin{align}
		||T(U)-T(V)||_\infty	&\leq	\left|\left|U^n-V^n\right|\right|_\infty\int\limits_0^\xi\int\limits_0^s \left(\frac{t}{s}\right)^{l-1}dtds\\
								&=		\left|\left|U^n-V^n\right|\right|_\infty\frac{\xi^2}{2l}
	\end{align}
	Let $n\geq1$. 
	Since the function $x\mapsto x^n$ is differentiable for $n\geq1$ it is also Lipschitz with constant $L$ on the compact interval $I=[0,\theta_0]$.
	This means in particular that we have
	\begin{equation}
		||T(U)-T(V)||_\infty	\leq	\left|\left|U-V\right|\right|_\infty\frac{L\xi^2}{2l}
	\end{equation}
	Thus when choosing $\epsilon<\sqrt{2m/L}$ we have a contraction and the Banach fixed-point theorem then gives us a unique solution to equation \ref{5-Zeroes-LEOperator}.\\
	Now let $0\leq n<1$.
	In this situation the preceding method does not work due to $x^n$ not being differentiable or Lipschitz at $x=0$. 
	Since by definition of $C^*$, we had chosen \eqref{5-Zeroes-LEOperator-Restrict-Slope} as an additional constraint, we can now choose $\epsilon$ small enough to have $\theta\geq q>0$ on $[0,\epsilon]$.
	Then the same procedure as before can be carried through but with the interval $I=[q,\theta]$.
\end{proof}
Equation \eqref{5-Zeroes-LEOperator-Derivative} shows that non-negative solutions must have non-positive slope.
The following theorem makes use of that fact and of a different representation of the \ac{LE} equation to show that solutions can be extended until they reach zero.
\begin{lemma}
	Let $\theta$ be a \ac{LE} solution on $[0,\epsilon)$. 
	Additionally suppose that $\lim\limits_{\xi\rightarrow\epsilon}\theta(\xi)>0$.
	Then we can find a $\delta$ such that an extension of $\theta$ around $\epsilon$ exists.
\end{lemma}
\begin{proof}
	First we will rewrite the \ac{LE} equation.
	\begin{align}
		\frac{d\theta}{d\xi} &= \chi\\
		\frac{d\chi}{d\xi} &= -\theta^n - \frac{2}{\xi}\chi
	\end{align}
	From here we can see that this equation at $\epsilon>0$ is solvable by means of the Picard-Lindelöf theorem.
	
\end{proof}
%
%
%
\subsection{TOV Equation}

%
%
%
