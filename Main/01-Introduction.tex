\section{Introduction}
\label{sec:01-Introduction}
The theory of \ac{gr} has allowed scientists deep insights into space and time itself.
From its emergence in 1916~\cite{einsteinGrundlageAllgemeinenRelativitaetstheorie1916a} to its current state it is regarded as one of the most elegant and well tested theories that Physics has to offer~\cite{hafeleAroundtheWorldAtomicClocks1972, vessotTestRelativisticGravitation1980a, battatApachePointObservatory2009}.
Most objects studied in this framework are typically very heavy or close to the speed of light and thus part of the large scale structure of our observable universe.
\ac{gr} is formulated in a geometric context by regarding space and time itself as a Lorentz-Manifold called spacetime with a non-semi-definite scalar product called the metric.
% This small difference has non-trivial implications when comparing Riemannian and Lorentzian geometry although we will not make any further effort to properly explain them in detail.
The curvature of this Lorentz-Manifold, which can be completely described by the Riemann tensor $R_{\mu \nu}{ }^{\sigma}{ }_{\lambda}$, depends on mass and energy present at the respective points in spacetime which is reflected in the Einstein equations~\cite{einsteinFeldgleichungenGravitation1915}.
\begin{equation}
	G_{\mu\nu}+\Lambda g_{\mu\nu} = R_{\mu\nu} + \left(\frac{1}{2}R+\Lambda\right)g_{\mu\nu}=8\pi T_{\mu\nu}
	\label{eq:01-Intr-Einstein-Equ}
\end{equation}
The left hand side of equation~\ref{eq:01-Intr-Einstein-Equ} contains the Einstein Tensor $G_{\mu\nu}=R_{\mu\nu}+1/2Rg_{\mu\nu}$ which is defined in terms of the Ricci $R_{\mu \nu}=R_{\lambda\mu}{}^{\lambda}{}_\nu$ and metric tensor $g_{\mu\nu}$ and the Ricci scalar $R=g^{\mu\nu}R_{\mu\nu}$.
The Energy Momentum tensor on the right-hand side carries information about mass and energy present at different points of space and time.
With the Levi-Civita connection on spacetime we can write the Riemann tensor as derivatives of the metric tensor which means the left-hand side of the Einstein equations~\ref{eq:01-Intr-Einstein-Equ} contains non-linear derivatives of $g_{\mu\nu}$.
% \begin{align}
%	\Gamma_{\mu\nu}^{\lambda} &= \frac{1}{2} g^{\lambda\sigma}\left(\partial_{\nu} g_{\sigma\mu}+\partial_{\mu} g_{\sigma\nu}-\partial_{\sigma} g_{\mu \nu}\right)
%	\label{eq:01-Intr-Riemann-Tensor-Levi-Civita-Metric-Derivatives-1}\\
%	R_{\mu \nu}{ }^{\sigma}{ }_{\lambda} &= \partial_{\mu} \Gamma_{\nu}^{\sigma}-\partial_{\nu} \Gamma_{\mu \lambda}^{\sigma}+\Gamma_{\mu \rho}^{\sigma} \Gamma_{\nu \lambda}^{\rho}-\Gamma_{\nu \rho}^{\sigma} \Gamma_{\mu \lambda}^{\rho}
%	\label{eq:01-Intr-Riemann-Tensor-Levi-Civita-Metric-Derivatives-2}
% \end{align}
This non-linearity is one of the key distinctions to other fundamental theories such as a Quantum Theory of Fields and is one of the reasons why few exact solutions of Einsteins equations exist.
In addition, the derivatives of $g_{\mu\nu}$ are also partial which means this theory in general has to deal with non-linear partial differential equations.
A behaviour that can not be altered by simply changing variables.
While these statements might sound demotivating, lots of methods to linearize or obtain \acp{ODE} have been developed.
In particular this thesis will use suitable chosen assumptions to derive a famous \ac{ODE}, the \ac{TOV} equation and analyse its properties.\\
Of the plentiful present problems one in particular of significant importance has been to calculate the macroscopic properties such as mass and angular momentum of stellar objects which in practice are often spherically or cylindrically symmetrical.
Stars, black holes, galaxies and galaxy clusters are just a few examples of this category.
The \ac{TOV} equation~\cite{tolmanStaticSolutionsEinstein1939, oppenheimerMassiveNeutronCores1939} relates pressure, density and mass inside a spherically symmetric object
\begin{equation}
	\begin{aligned}
			\frac{\partial m}{\partial r} &= &&4\pi\rho(r)r^2\\
			\frac{\partial p}{\partial r} &= -&&\frac{Gm\rho}{r^2}\left(1+\frac{p}{\rho c^2}\right)\left(\frac{4\pi r^3 p}{mc^2}+1\right)\left(1-\frac{2Gm}{rc^2}\right)^{-1}
			\label{eq:01-Intr-TOV-Equation}
	\end{aligned}
\end{equation}
and given an \ac{eos} $\rho(p)$ one can solve the differential equation and thus obtain parameters such as the mass and total radius of said stellar object.
Together with non-negative initial values, we specifically prove, that solutions in general exist and are unique.
The non-relativisitc limit of the \ac{TOV} equation is the \ac{LE} equation~\cite{laneTheoreticalTemperatureSun1870, emdenGaskugeln1907} which can be derived in the Newtonian theory of motion when applying the condition of hydrostatic equilibrium.
% \begin{equation}
%	\frac{\partial \Phi}{\partial r} = - \frac{1}{\rho}\frac{\partial p}{\partial r}
%	\label{eq:01-Intr-Hydrostatic-Equilibrium}
% \end{equation}
The outside solution of the \ac{TOV} equation is characterised by the well known Schwarzschild solution~\cite{schwarzschildUberGravitationsfeldMassenpunktes1916} which is also the outside solution of black holes
\begin{equation}
	g=-\left(1-\frac{r_{\mathrm{s}}}{r}\right) c^{2} \diff t^{2}+\left(1-\frac{r_{\mathrm{s}}}{r}\right)^{-1} \diff r^{2}+r^{2}\left( \diff\theta^2+\sin^2\theta \diff\phi^2\right)
	\label{eq:01-Intr-Schwarzschild-Metric}
\end{equation}
and where $r_\mathrm{s}=2GM/c^2$ is the Schwarzschild radius.
Numerical results demonstrate that the behaviour of zero values for the \ac{TOV} equation is similar to that of the \ac{LE} equation, details of which will be discussed in section~\ref{subsec:4-NumSol-Sec-TOV-Exponents}.\\
% As was explained just before, an additional piece of information, the \ac{eos} needs to added to the \ac{TOV} equation in order to obtain solutions.
From a Physics perspective, it is clear that Thermodynamics has to play a role in the description of stars since one has to deal with large numbers of particles well in equilibrium.
Thermodynamics is a statistical theory of nature and describes macroscopic phenomena by knowing microscopic behaviour of particles.
Carnot~\cite{carnotReflexionsPuissanceMotrice1824} outlined the first relations between a thermodynamic processes an engine and motive power.
Following in his steps many researchers such as Thomson (Lord Kelvin)~\cite{thomsonABSOLUTETHERMOMETRICSCALE2011}, Clausius~\cite{clausiusMechanischeWaermetheorie1876}, Maxwell~\cite{maxwellScientificLettersPapers2002}, Boltzmann~\cite{boltzmannUberMechanischeBedeutung1866} and Gibbs~\cite{gibbsElementaryPrinciplesStatistical2010} greatly impacted the development of Thermodynamics.
Over the years, different ensembles have emerged which can equivalently, although in practice often simplified by a specific choice, describe the same physical system.
The canonical ensemble was first described by Ludwig Boltzmann~\cite{boltzmannUeberEigenschaftenMonocyclischer1885a}.
Its corresponding partition function is given by
\begin{equation}
	\mathcal{Z}(T,V,N) = \int\exp\left(-\frac{H(x_1,\dots,p_N)}{k_B T}\right)\frac{\diff x_1\dots \diff p_N}{N!h^{3N}}
	\label{eq:01-Intr-Canocnical-Ens-Part-Funct}
\end{equation}
where $H$ denotes the Hamiltonian of the particles.
The partition function $\mathcal{Z}$ can then be used to calculate properties such as pressure $p$, internal energy $\mathcal{U}$ and thus also the energy density $\mathcal{U}/V$.
Together with an adiabatic condition one would hope to find a relation between energy density and pressure to derive a \acp{eos}.
In many examples, a polytropic \ac{eos} between pressure $p$ and density $\rho$ is assumed
\begin{equation}
	\rho = Ap^{1+1/n}
	\label{eq:01-Intr-Poly-EOS}
\end{equation}
where $A$ is a constant.
The exponent $n=1/(\gamma-1)$ can in a special case be related to the adiabatic index $\gamma$ of a thermodynamic process.
In this case, equation~\ref{eq:01-Intr-Poly-EOS} is derived using the ideal gas equation $pV=Nk_B T$ and adiabaticity $\updelta Q=0$.
Another important relation is between $\gamma$ and degrees of freedom.
For an ideal gas, the index can be given by $\gamma=1+2/f$, where $f$ is the degrees of freedom.
In this case, $\gamma$ directly yields information about the underlying microscopic behaviour of the gas.
While there exist more recent examples of more complex \ac{eos} models~\cite{hummerEquationStateStellar1988}, this equation already covers a wide range of cases~\cite{horedtPolytropesApplicationsAstrophysics2004}.
The first part of this thesis will be concerned with deriving a fully special relativistic \ac{eos} of a non-interacting gas that will later be used in numerical solutions of the \ac{TOV} equation.