\begin{section}{More Exact Results}
\begin{subsection}{LE equation}
One can derive another exact solution for the LE equation at the exponent $n=2$ when 
considering a series expansion of $\theta$ around the point $\xi=0$ with initial conditions
\begin{equation}
	\theta=\sum\limits_{m=0}^\infty a_m\xi^m \hspace{1cm} a_0=\left.\theta\right|_{\xi=0}=\theta_0 
	\hspace{1cm} a_1=\left.\frac{d\theta}{d\xi}\right|_{\xi=0}=0
	\label{5-MoExSo-LEN2-Series-Initial-Conditions}
\end{equation}
Since the series is absolut convergent, we can plug it into the LE equation 
\ref{3-Mass-Lane-Emden-Eq} and use the Cauchy Product formula.
\begin{equation}
	\sum\limits_{m=2}^\infty m(m-1)a_m\xi^{m-2}+\sum\limits_{m=1}^\infty (2m)a_m\xi^{m-2} + 
	\sum\limits_{m=0}^\infty\sum\limits_{k=0}^m a_{m-k}a_k\xi^m = 0
	\label{5-MoExSo-LEN2-Series-PluggedIn}
\end{equation}
\begin{theorem}
	The odd coefficients $a_{2m+1}$ of this series expansion vanish.
\end{theorem}
\begin{proof}
	We rewrite the summations of equation \ref{5-MoExSo-LEN2-Series-PluggedIn} to start at the 
	same index $m=0$ and combine them
	\begin{equation}
		\sum\limits_{m=0}^\infty\left((m+2)(m+1)a_{m+2}\xi^{m}+(2m+2)a_{m+1}\xi^{m-1} + 
		\sum\limits_{k=0}^m a_{m-k}a_k\xi^m\right) = 0
	\end{equation}
	With equation \ref{5-MoExSo-LEN2-Series-Initial-Conditions}, we can start the summation 
	in the middle one index higher and separate the term $\xi^m$. This equation has to be true 
	inside the radius of convergence of the series and thus needs to vanish for ambiguous $\xi$.
	\begin{equation}
		(m+2)(m+1)a_{m+2}+2(m+2)a_{m+2}+\sum\limits_{k=0}^ma_{m-k}a_k = 0
	\end{equation}
	and upon further manipulation results in the recursive description for the coefficients 
	of the series
	\begin{equation}
		a_{m+2} = -\frac{1}{(m+2)(m+3)}\sum\limits_{k=0}^ma_{m-k}a_k.
		\label{5-MoExSo-LEN2-Recursive-Coefficients}
	\end{equation}
	We show the statement by induction. For $a_1$ it is already true. Let the statement be true 
	for all odd values $2k+1\leq2m+1$. Writing down $a_{2m+3}$ gives us
	\begin{equation}
		a_{2m+3} = -\frac{1}{(2m+3)(2m+4)}\left(a_0a_{2m+1}+a_1a_{2m}+\dots+a_{2m}a_1+a_{2m+1}a_0\right).
	\end{equation}
	It is clear that in the summation odd and even coefficients get paired and will thus 
	vanish completely.
\end{proof}\noindent
This proof shows that we can restrict ourselves to the subsequence $b_m=a_{2m}$ and 
the subseries with initial values given by
\begin{equation}
	\theta = \sum\limits_{m=0}^\infty b_m\xi^{2m} \hspace{1cm} b_{m+1} = 
	-\frac{1}{(2m+2)(2m+3)}\sum\limits_{k=0}^mb_{m-k}b_k \hspace{1cm} b_0=\theta_0
	\label{5-MoExSo-LEN2-bn-Definition}
\end{equation}
\begin{theorem}
	The series $\theta=\sum\limits_{m=0}^\infty b_m\xi^{2m}$ converges for $\xi\leq1$.
\end{theorem}
\begin{proof}
	We start by showing that $|b_{m+1}|\leq1/(4m+6)$. Using the triangle inequality on 
	\ref{5-MoExSo-LEN2-bn-Definition}, we obtain
	\begin{equation}
		|b_{m+1}| \leq \frac{1}{(2m+2)(2m+3)}\sum\limits_{k=0}^m|b_{m-k}b_k|.
	\end{equation}
	For $m=0$, we have $b_1=-1/6$ and thus $|b_1|\leq1/3$ and so the statement is true for $m=0$.
	If the statement holds for all $m<m+1$, we have
	\begin{equation}
		|b_{m+1}| \leq \frac{m}{(2m+2)(2m+3)} = \frac{1}{2}\frac{m}{(m+1)}\frac{1}{(2m+3)}
		\leq\frac{1}{2}\frac{1}{(2m+3)}
	\end{equation}
	which proves the first statement. To see that $b_m$ is a alternating sequence, we again inspect
	\ref{5-MoExSo-LEN2-bn-Definition}. The first element of the series is 
	$b_0=1$ and the second is $b_1=-1/6$ so the initial statement is again correct. Let $b_m$ be 
	alternating for all $m<2m+1$. Then equation \ref{5-MoExSo-LEN2-bn-Definition} for odd values reads
	\begin{equation}
		b_{2m+1} = -\frac{1}{(4m+4)(4m+5)}\left(b_{2m}b_0+b_{2m-1}b_1+\dots+b_0b_{2m}\right)
	\end{equation}
	This shows that if all odd values $b_{2k+1}$ are negative and all even ones are positive, 
	then $b_{2m+1}$ is negative too. The same holds true when the index $2m+2$ is even.
	By the Leibniz criterion the series converges if $\xi\leq1$. 
\end{proof}
\begin{figure}[H]
	\import{pictures/5-MoreExactSolutions/}{LE-ExactN2.pgf}
	\caption[LE Solution for $n=2$]{LE Solution for $n=2$ - We see the power 
	the series calculated with the coefficients explained above (the far right tick is the 
	last calculated value for $R_m$ where plotting is stopped) and 
	the sequence $R_m=(|a_m|)^{-1/m}$ which approaches the radius of convergence.
	}
	\label{5-MoExSo-LEN2-Plot}
\end{figure}\noindent
These results can be used to calculate the power series of the LE equation numerically. 
Figure \ref{5-MoExSo-LEN2-Plot} shows those results. Good agreement is found between the two solutions 
as the difference $\Delta=\theta_{calc}-\theta_{ser}$ is not larger than $0.00015$.
% However in both plots at the bottom, the last few values where the solution starts reaching the 
% radius of convergence have been omitted for visual clarity. 
The solutions differ for larger values around $\approx3.918$
The top right plot indicates that the radius of convergence might actually be larger than 
where solutions deviate. 
One important observation is that in this procedure, 
the coefficients were calculated up to $b_{250}\approx-4.7723\times10^{-296}$ which is in close 
proximity to the floating point limit of python given by $\approx2.2251\times10^{-308}$ and 
indicates that numerical errors may be the reason the series starts to diverge. 
Also in order to achieve sufficient numerical precision, we would need to calculate higher 
order terms of the series, which is supported by $(R_{250})^{500}\approx3.2159\times10^{296}$ and shows
that these parts of the series still yield large contributions to the value of $\theta$ at this point.



\end{subsection}
%
%
\begin{subsection}{TOV equation}

\end{subsection}
%
%
\begin{subsection}{Existance of Solutions}

\end{subsection}
\end{section}
