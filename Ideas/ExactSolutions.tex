\begin{section}{Exact Solutions - Lane Emden}
\begin{subsection}{Special Cases}
The Lane-Emden equation \ref{3-Lane-Emden-Eq} with initial values $\theta(0)=1$ and $\partial_x\theta(0)=0$ has explicit solutions \cite[p.~91]{chandrasekharChandrasekharAnIntroductionStudy1958} for the special values of $n=0,1,5$.
\begin{align}
	\theta_0(\xi) &= 1-\frac{1}{6}\xi^2\\
	\theta_1(\xi) &= \frac{\sin(\xi)}{x}\\
	\theta_5(\xi) &= \sqrt{1+\frac{1}{3}\xi^2}^{-1}
\end{align}
The solutions for $n=0,1$ range from $\xi=0$ to a zero point at $\xi_0=\sqrt{6},2\pi$ respectively. The solution for $n=5$ does not exhibit such behaviour. However for $n<5$ numerical results \cite{pleyerGithubRepositoryJonas} that can be seen in figure \ref{5-ExaSol-LE-Exponents} indicate that the behaviour for $n=0,1$ is upheld and only fades once $n=5$.
\begin{figure}[ht]
	\centering
	\includesvg[width=\textwidth]{pictures/5-ExactSolutions/LE-Exponents}
	\caption[Lane Emden Exponents]{This figure shows the value $\xi_0$ at which $\theta(\xi_0)=0$. As can be seen, the value grows exponentially when approaching the exponent $n=5$.}
	\label{5-ExaSol-LE-Exponents}
\end{figure}

\end{subsection}
%
%
%
\begin{subsection}{Existance of Solutions}
We begin by transforming the differential equation with the substitution $\xi=\exp(y)$ and with $d\xi = \exp(y)dy$ obtain
\begin{equation}
	\frac{d^2\theta}{dy^2} + \frac{d\theta}{dy} + \exp(y)\theta^n=0
\end{equation}
or expressed as a system of first order differential equations
\begin{align}
	\frac{d\chi}{dy} &= -\exp(y)\theta^n-\chi\nonumber\\
	\frac{d\theta}{dy} &= \chi
\end{align}

\end{subsection}


\end{section}
