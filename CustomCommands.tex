% Symbols for Number sets and other stuff
\newcommand{\Z}{\mathcal{Z}}
\newcommand{\F}{\mathcal{F}}
\newcommand{\Ent}{\mathcal{S}}
\newcommand{\U}{\mathcal{U}}
\newcommand{\R}{\mathbb{R}}
\newcommand{\e}{\mathit{e}}
\newcommand{\RP}{\mathbb{R}\text{P}}

% Declare this - symbol to import pgf files correctly
\DeclareUnicodeCharacter{2212}{-}

\newtheoremstyle{defstyle}
	{\topsep} 											% ABOVESPACE
	{\topsep} 											% BELOWSPACE
	{}													% BODYFONT
	{}													% INDENT (empty value is the same as 0pt)
	{\bfseries}{}										% HEADFONT
	{\newline}											% HEADPUNCT
	{\thmname{#1}~\thmnumber{#2}\thmnote{\ -\ #3}}%		% CUSTOM HEAD SPEC

\newtheoremstyle{normalstyle}
	{\topsep} 											% ABOVESPACE
	{\topsep} 											% BELOWSPACE
	{}													% BODYFONT
	{}													% INDENT (empty value is the same as 0pt)
	{\bfseries}{}										% HEADFONT
	{\newline}											% HEADPUNCT
	{\thmname{#1}~\thmnumber{#2}\thmnote{\ -\ #3}}%		% CUSTOM HEAD SPEC

\newtheoremstyle{remarkstyle}
	{\topsep} 											% ABOVESPACE
	{\topsep} 											% BELOWSPACE
	{}													% BODYFONT
	{}													% INDENT (empty value is the same as 0pt)
	{\itshape}{}										% HEADFONT
	{\newline}											% HEADPUNCT
	{\thmname{#1}~\thmnumber{#2}\thmnote{\ -\ #3}}%		% CUSTOM HEAD SPEC

\newtheoremstyle{postulatestyle}
	{\topsep} 											% ABOVESPACE
	{\topsep} 											% BELOWSPACE
	{}													% BODYFONT
	{}													% INDENT (empty value is the same as 0pt)
	{\bfseries}{}										% HEADFONT
	{\newline}											% HEADPUNCT
	{\thmname{#1}\thmnote{ #3}}%		% CUSTOM HEAD SPEC
	
%%%%%%%%%%%%%%%%%%%%%%%%%%%%%%%%%%%%%%%%%%%%%%%%%%%%%%%%%%%%%%%%%%%%%%%%%%%%%
% Define how Theorems and Stuff work
%%%%%%%%%%%%%%%%%%%%%%%%%%%%%%%%%%%%%%%%%%%%%%%%%%%%%%%%%%%%%%%%%%%%%%%%%%%%%

% Get rid of points after theorem names.
% \makeatletter
% \g@addto@macro\th@defstyle{\thm@headpunct{:}}
% \makeatother

% Define different theorem environments.
\theoremstyle{normalstyle}
\newtheorem{theorem}{Theorem}[section]
\newtheorem{corollary}[theorem]{Corollary}
\newtheorem{lemma}[theorem]{Lemma}
\newtheorem{example}[theorem]{Example}

\theoremstyle{remarkstyle}
\newtheorem*{remark}{Remark}

\theoremstyle{defstyle}
\newtheorem{definition}[theorem]{Definition}
\theoremstyle{postulatestyle}
\newtheorem*{postulate}{Postulate}

% Define a different numbering for enumerate environments
\renewcommand{\labelenumi}{\roman{enumi})}
